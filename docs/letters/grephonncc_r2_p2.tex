\documentclass[11pt]{article}
\usepackage[top=1.00in, bottom=1.0in, left=1.1in, right=1.1in]{geometry}
\usepackage{graphicx}
\usepackage{natbib}
\usepackage{amsmath}
\usepackage{lineno}
\usepackage{xr-hyper}
\externaldocument{..//grephonms}
\newcommand{\R}[1]{\label{#1}\linelabel{#1}}
\newcommand{\lr}[1]{line~\lineref{#1}}
\usepackage{hyperref}
\setlength\parindent{0pt}


\begin{document}
\setlength{\parindent}{0cm}
\setlength{\parskip}{7pt}

\bibliographystyle{/Users/Lizzie/Documents/EndnoteRelated/Bibtex/styles/besjournals}
\lrenewcommand{\lrefname}{\CHead{}}

Reviewer comments (we provide below the full context of the reviewer's comments) are in \emph{italics}, while our responses are in regular text. \\ 

{\bf Reviewer comments:} \\

{\bf Reviewer: 2}

\emph{I have re-evaluated the revised submission by Wolkovich et al. and appreciate the effort the authors have put into addressing my previous concerns. While the authors have made progress by updating statements and expanding discussions, I remain uncertain whether the proposed framework can effectively guide readers in examining the conditions under which longer seasons will---or will not---lead to greater growth.}\\

We thank the reviewer for their continuing efforts to help improve this manuscript to make it a useful contribution to the field. Based on their feedback, we have revised the manuscript in a number of places as detailed below. \\

\emph{The authors review the individual hypotheses proposed by previous studies, and attempt to group them in order to clarify the possible mechanisms controlling the relationships between season length and growth. This effort is crucial for enhancing our understanding of how changes in season length drive growth and for improving our predictive capabilities. However, these mechanisms have already been well established, and readers may find them familiar, that’s what I’m feeling when I read the manuscript and the revision. While existing studies have largely focused on testing these individual hypotheses, the driving effect of growing season length on growth is influenced by many factors. For example, changes in precipitation regimes, especially the extreme events, can also alter the relationship between season length and growth. As a review paper, outlining all possible mechanisms that could affect these relationships is important for readers who wish to test them in future research. Unfortunately, this part of the content is currently not complete coverage.}\\

We agree that we should do more here to push the field beyond the current hypotheses and approaches. To address this we have made a number of changes throughout. We have changed the tone of the manuscript to clarify we do not consider this a review paper, but more a way to evaluate the current state of the field and push it forward, with changes to the abstract, introduction (\lr{fornccS}-\lr{fornccE}) and in later sections as we outline our suggested paths forward (e.g., \lr{notallmech}-\lr{morebox1}). Given the length and aims of this manuscript, we do not feel we can cover all mechanisms and have worked to clarify that (e.g., \lr{notallmech}, \lr{notallmech2}, \lr{forbigKref1}), but we agree that extreme events should have been covered explicitly and now include them (\lr{bigKextremeS}-\lr{bigKextremeE}).\\


\emph{Importantly, the effects of diverse factors on plant growth are highly complex, and the driving effect of growing season length on growth is influenced by many factors with significant interactions among predictors. For example, climatic warming can shift vegetation phenophases through intrinsic developmental processes, affecting season length and regulating tree growth (please refer to the review article titled ``Shifts in Plant Growth Control''). We still lack a comprehensive mechanistic framework to elucidate how external and internal drivers jointly regulate this relationship. Such a mechanistic framework is crucial for clarifying the conditions under which longer seasons will---or will not---lead to greater growth. Importantly, this framework can guide readers in assessing how changes in internal constraints influence tree growth under climate change, thereby unraveling the complex interactions that drive growth. In this revised version, the authors have indeeded provided a list of tasks to be addressed and have suggested specific types of experiments. I appreciate the effort and dedication the authors have invested, especially the phylogentic issues, but in my opinion as a review paper, the absence of a comprehensive mechanistic framework reduce the impact of the article.}\\

We understand the reviewer's concern and agree that the lack of comprehensive mechanistic physiological model for growth is a major problem for the field and one we tried to point out in previous versions of this manuscript though we now see it was not clear. We personally think our paper includes major ways to address this problem, including important new perspectives, but we now do more to clarify that the lack of a comprehensive model is a critical hurdle (\lr{fornccS}, \lr{forbigKnomodel1} and \lr{forbigKref2S}-\lr{forbigKref2E}) and we cite a number of papers that have worked towards this comprehensive model (e.g., \lr{forbigKref0}, \lr{forbigKref1}, \lr{forbigKref2E}). Given these papers we feel including another possible model in this manuscript may be less useful than highlighting the gap and progress towards it.

We completely agree with the reviewer that how internal and external factors interact is an important area. We mentioned this previously as one of the major open questions, but we can see now that we effectively under-emphasized something we see as one of the biggest problems preventing progress. To address this, we now stress throughout the manuscript that understanding these interactions is critical (e.g., \lr{forbigKref1}, \lr{forbigKinteract1}, \lr{bigKextremeE}) in addition to our subsection, `How do external drivers and internal constraints interact?'\\

\emph{The authors' proposal of a trait-based phylogenetic model represents a novel approach, providing a natural framework to organize species' (and potentially populations', which nest within species) responses to longer seasons. Additionally, the authors offer guidance on the next steps in methodology. However, a significant gap remains in our mechanistic understanding of the relationship between season length and growth, which may limit the model's predictive accuracy. For instance, climate change not only affects season length but also alters vegetation growth rates, thereby influencing overall growth. The authors' review of existing literature has highlighted the importance of growth rates in regulating tree growth (see supplementary References: Colangelo et al., 2022; Ren et al., 2019). Recent studies have also identified optimal growth temperatures that affect plant growth processes and have incorporated these factors into models to enhance predictive accuracy. These are important issues that are defintivel need to be coupled. As a review paper, the reader would expect some specific directions for future research to enhance our mechanistic understanding of this relationship, and improving the predictions of the models. For example, exploring how external environmental factors influence internal growth processes and, in turn, affect the relationship between season length and growth, as well as how to conduct such research.}\\

We appreciate the reviewer's positive comments on our trait-based phylogenetic approach and guidance on next steps, and we agree with the reviewer on the gaps they outline. In addition to changes outlined above to clarify the goals and limitations of our manuscript, as well as the critical need for a more comprehensive mechanistic physiological model, we have added more on the need to better incorporate temperature, moisture, and nutrient constraints (\lr{forbigKNUTSetcS}-\lr{forbigKNUTSetcE}) and updated the related figure. We have also extended our discussion of designing studies to better study constaints (\lr{moconstrainS}-\lr{moconstrainE}) and their interactions with external drivers through better networks (\lr{bigKextreme2}).\\

\emph{In summary, while the topic is important, and the lack of a comprehensive mechanistic framework and a clear direction for future researches, that is likely due to currently limited inversitigations in this topic, makes this paper feel somewhat premature.}\\

We understand these concerns, and believe our revisions have better framed the state of the field, the aims of our manuscript, and provide an important contribution to move the field forward.\\


\end{document}


\newpage
{\bf References:}
\bibliography{/Users/Lizzie/Documents/git/bibtex/LizzieMainMinimal}

