\documentclass[11pt]{article}
\usepackage[top=1.00in, bottom=1.0in, left=1.1in, right=1.1in]{geometry}
\usepackage{graphicx}
\usepackage{natbib}
\usepackage{amsmath}
\usepackage{lineno}
\usepackage{xr-hyper}
\externaldocument{..//grephonms}
\newcommand{\R}[1]{\label{#1}\linelabel{#1}}
\newcommand{\lr}[1]{line~\lineref{#1}}
\usepackage{hyperref}
\setlength\parindent{0pt}


\begin{document}
\setlength{\parindent}{0cm}
\setlength{\parskip}{7pt}

\bibliographystyle{/Users/Lizzie/Documents/EndnoteRelated/Bibtex/styles/besjournals}
\renewcommand{\refname}{\CHead{}}

Editor and reviewer comments (we provide below the full context of the three reviewers' comments) are in \emph{italics}, while our responses are in regular text. \\ 

{\bf Response to editors' comments:} \\  % See dothisGREPHON for notes on figures and overhaul of Section 3. 

We appreciate the editors' thoughtful comments and the chance to revise this work. ...

\begin{enumerate}
\item Completely overhauled the third section (`Building a new framework for growth × season length relationships') to be focused on the most critical questions to answer and clear suggestions of how to do it
\item Added a brief discussion of primary versus secondary growth on lines ...  and we return this as we develop our points  (concern of R1 and R3) 
\item New figure and clear discussion of importance of understanding growth rate (concern of R2) 
\item Added new figures ... 
\end{enumerate}

Overlapping concerns across reviewers:
\begin{enumerate}
\item Growth rate (err, maybe just R2) 
\item Different types of growth (primary versus secondary ... roots?):  R1 and R3
\item No clear framework! Not sure what to do next. R2 and R3 
\end{enumerate}

{\bf Reviewer comments:} 

{\bf Reviewer: 1}


\emph{In this manuscript, the authors address the critical question of whether the lengthening of the growing season directly translates into enhanced growth performances. The topic is highly relevant, especially given its implications for climate change predictions and carbon sequestration models. The authors cover a wide range of perspectives, from eco-physiological processes to the implications of predictive modeling, providing a comprehensive review of the current literature. Overall, this work is a thorough and well-organized examination of the complexities surrounding this topic, and the authors did excellent work in synthesizing diverse research findings. I believe the manuscript is suitable for publication, though I suggest that the authors further develop the following two key areas:}


\emph{1) While the manuscript touches on the general concept of growth, it would benefit from a clearer differentiation between primary and secondary growth. Primary and secondary growth are processes with different temporal and environmental dependencies. In particular, primary growth is influenced by environmental conditions from previous growing seasons, while secondary growth is directly and exclusively responsive to current-year conditions. The manuscript discusses carbon allocation investments (e.g., on page 6, the competition between growth and reproduction). Still, it would be helpful to explicitly link these discussions to how primary and secondary growth might compete for resources differently. In this framework, I wonder if the hypothesis that a lengthened growing season might affect these two growth processes differently should be clearly raised, as there is a significant gap in the current understanding. Some recent studies suggest that extended growing seasons may enhance primary growth more than secondary growth due to the timing of carbon allocation and physiological constraints. This aspect deserves a deeper discussion, as it could affect future predictions of carbon storage.}

We agree ... now mention it ... and added a new figure!\\


\emph{2) The authors rightly mention that the definition of "growing season" is debated, and this debate plays a critical role in understanding the relationship between season length and growth. To date, studies have varied significantly in their definitions, with some focusing on temperature thresholds while others emphasize the onset and cessation of physiological activity. This discrepancy complicates comparisons across studies. A more detailed explanation of the different definitions and their implications would greatly enhance manuscript accessibility, particularly for readers less familiar with the concept of phenology. Additionally, since the growing season length is a critical variable in models forecasting carbon sequestration, understanding its definition is crucial for interpreting results. I recommend a dedicated section that elaborates on the current definitions, the ongoing debate, and how different interpretations could lead to divergent conclusions about tree growth under climate change.}

Agreed, we added a new figure an refer readers to a longer discussion of this now in the supplement.\\

{\bf Reviewer: 2}


\emph{This study reviews the recent literature (36 papers) to group the changes in plant growth with season length, and finds that 58\% of studies supported the assumption of increased growth with longer season, while 36\% of studies did not. Then, the authors group the pathways or hypotheses that how climate change alters growing season length and then affects growth from a review of papers studying growth × growing season length. Finally, they outline how bridging these current divides while simultaneously integrating ecological theory could yield new advances in fundamental biology. The topic of “why longer seasons with climate change may not increase tree growth” is important and interesting. However, this review only groups some possible reasons for the relationship between growth and season length from a review of papers, and the underlying mechanisms remain unclear and the discussion also limit with future directions in this topic.}


\emph{Major concerns:}\\
\emph{1. The possible reasons (pathways or hypotheses) that the authors grouped from the papers studying growth × growing season length are single, and have been well-know in previous study. However, the effects of diverse factors on plant growth are complex. This review does not give an overall framework for how changes in plant growth with the growth season via external drivers and internal constraints. For example, we don’t know how/whether climatic factors directly or indirectly alter growth seasons via internal constraints, and then affect growth. This is a key point to disengage and understanding when longer seasons will—or will not—lead to greater growth.}

We totally agree! We overhauled the last section and now emphasize this in `How do external drivers and internal constraints act together?'

The review endeavors to group papers by the individual hypotheses they test. We are not saying that these hypotheses are mutually exclusive, we agree with the reviewer that the reality is far more complicated than any individual hypothesis here can express. We have added a clarifying statement at LINE  ... %% 

 In revising our manuscript, we have increased the emphasis on what we need to know and future directions for research and modelling. We hope that focusing on how very much we still need to resolve how climatic factors shape growth, we will encourage an increase in mechanism-centered experimental design and model development.\\


\emph{2. As shown in Fig. 1d-e in the manuscript, climatic factors alter plant’s physiological process (i.e., growth rate) rather than growth season length, and then dominate plant growth. This indicates that the physiological process plays a crucial role in regulating growth. In other words, if climate change reduces the growth rate, a longer growth season may induce higher, unchanged, or lower plant growth. In contrast, if climate change shortens the growth seasons, a higher growth rate may induce higher, unchanged, or lower plant growth. However, this review ignores the importance of growth rate, although some papers that they reviewed have mentioned the growth rate (see supp References: Colangelo et al., 2022; Zhang et al., 2021; Ren et al., 2019).}

We maybe add a new figure (moraconcept)? \\

The reviewer makes a good point that growth rate is an important aspect of the interaction of total growth with climate and ‘season length’. For simplicity, our review attempts to focus entirely on studies looking at growing season-length. While it is true that some of the paper we cite consider rate, their shared characteristic is linking phenology to total annual growth. 

However, the influence of climate on growth rate is an important part of unraveling the mechanisms underlying realized annual growth. In our expanded discussion of future directions, we have placed growth rate as an important target metric for increasing study resolution at LINE ...%% 


\emph{3. As shown in Fig. 1g in the manuscript, this hypothesis indicates that the growing season length and growth are unchanged due to internal constraints. However, this hypothesis may be wrong and is not supported by the listed References, (e.g., Zohner et al., 2023 SCIENCE). 1) the growth onset is not unchanged, and should be advanced under global warming. For example, recent research has suggested that warming-induced earlier spring phenology is driven by increased photosynthetic carbon assimilation in the previous growing season (see Gu et al., 2022 NATURE COMMUNICATIONS). 2) The peak growth and their timing should be increased and advanced due to source-sink balances/limitations under global change (see Korner et al., 2015; Green et al., 2022 SCIENCE). 3) the growth end timing is also changed and is jointly controlled by climatic factors and internal constraints (see Zohner et al., 2023 SCIENCE).}

We are not advocating the hypotheses in Fig. 1, indeed our review indicates that the impacts of growing season length on growth are highly dependent on progeny, species, location, method, and tree age. None of them are correct for every situation, they simply reflect what is currently being tested experimentally. % Alana says: Okay, so 95\% sure the reviewer thinks we are advocating something here. Maybe the fig caption needs a statement of neutrality


\emph{4. The authors state that “this research builds a mechanistic framework for when longer seasons will—or will not—lead to greater growth, with major forecasting implications”. However, after carefully reading this paper, I still struggle with how to examine and explore when longer seasons will—or will not—lead to greater growth. As described by above comment#1, the mechanistic framework and the underlying mechanisms remain unclear.}

Yeah, R3 said the same thing so we overhauled the whole damn end and you gotta admit -- we have some new ideas! But you might not like them ...\\

\emph{5. For the suggestions for future work, the authors state that “Using existing data and networks to partition levels of variation across drivers, New experiments to tease apart external & internal drivers, Models that push forward theory and aid forecasting”. However, these suggestions are methods, and I still don't know exactly how/what to do next. For example, how to disentangle the changes in vegetation phenophases and season length by external drivers and internal constraints. This is important to improve our understanding of the relationship between season length and plant growth, and our prediction accuracy of the model. Therefore, this review should give what work have we done, and how/what work needs to be done next.}


{\bf Reviewer: 3}

\emph{1. LIMITED NOVELTY OF THE ECOLOGICAL DISCUSSION AND THE SUGGESTED FRAMEWORK}


\emph{The manuscript provides a nice and comprehensive discussion about the ecological phenomena underlying the potential relationship between growing season length and tree growth. That discussion then leads logically to the suggested mechanistic framework for subsequent studies. Both parts read well, but at the end I was wondering, how much really novel aspects were provided. When reading the text I got the feeling that yes, I have heard this before: we need to understand all of these ecological phenomena better, if we are supposed to understand and project the development and trees and forests under climatic change.}


\emph{Similarly, in my view, the novelty of the suggested framework is limited. For instance, it has been known all too well for a long time that the mechanistic and the statistical models both have their strengths and weaknesses and that the two model categories should be better linked to each other. But how to do that? In other words, the discussion concerning the framework, too, reads well, but it lacks concrete novel aspects. One concrete suggestion is though made by recommending the use of the two major databases ITRDB and PEP725 simultaneously (Fig. 5). However, in my view it would be better to realize the idea in a separate study, not just to mention the idea here in a long list of to-do’s. And yes, integrative approaches over scales and study disciplines, and a better integration of ecological theory, are all definitely needed. But we have heard most, if not all, of that, too, already before.}

Noted! And R2 complained also, so we overhauled the last third and streamlined the second section. Hopefully you're less bored now?\\

\emph{In brief: to my understanding the suggested framework is a to-do list which is not especially novel. However, the authors may disagree with me about this. If they do, then I recommend that they point out the truly novel aspects more clearly. Now the possible novel aspects are hidden in the textbook- like text.}

We think it's less boring now!\\

\emph{Even in its present form the suggested framework would fulfil the requirements of most journals. However, I doubt if the framework fulfils the rigorous requirements of NCC.}


\emph{2. DIFFERENT ASPECTS OF GROWTH SHOULS BE DISCUSSED}

\emph{Unless I am mistaken, photosynthesis is taken in the study as one metric of growth. I can see the motivation for that, but because that is contradictory to the established use of these concepts in plant ecophysiology and ecology, this issue should be elaborated. A failure to do that would not be a good start for the integrative approaches the authors are calling for.}

Repeat what we say to R1 but in different words.\\

\emph{Along the same line of reasoning, different aspect of growth as such should be better elaborated and discussed. The discussion on determinate species on page 6, for instance, is relevant only for height growth, not for diameter growth, or accumulation of biomass. Because of these reasons, a general theory between growing season length and ‘growth’ maybe even impossible, not only difficult to achieve. In my view, this overall limitation should be addressed more clearly in the study. What exactly is aimed at in this and the related studies? If the aim is to examine carbon sequestration, then the discussion on determinate species is not relevant. Those species fix carbon long after their height growth cessation has taken place.}

\emph{3. UNCLEAR DOCUMENTATION OF METHODS}

\emph{The manuscript lacks a Method-section, so the methods are documented only in the supplementary material. Whether that is allowed or not in NCC is up to the editors, but more seriously, the documentation of the methods needs to be edited for better clarity. Most importantly, the method documentation should tell as clearly as possible the rules, how a given examined published paper found its place in Fig. 1. I read the method documentation several times, but this crucial issue remained largely unclear to me. I understand that given the diversity inherent in the original studies examined, documentation of the rules for the classifications done in the present study is not an easy task. However, methods need to be properly documented also when the documentation is not easy.}

We should fix the supp.... can someone else take a first stab? \\

We have included a section in \emph{Literature review methods} of the Supplement to better document our methods for assessing hypothesis testing across the review. 

\emph{4. MISCELLANEOUS COMMENTS}

\emph{Figures and their legends should be as clear and as self-explanatory as possible. Regarding this, the crucial Fig. 1 and its legend needs a major editing. The two panels on the right are clear and informative, but almost everything else needs clarification and elaboration. The symbols need to be explained, better contrasting colours should be used for Pre-CC and Post-CC, etc. An intelligent reader perhaps is able to grasp everything in the present figure quickly, but the figures should be as easy-to-read as possible, not a test for the intelligence of the reader.}

\emph{Figures and their legends should be as clear and as self-explanatory as possible. Regarding this, the crucial Fig. 1 and its legend needs a major editing. The two panels on the right are clear and informative, but almost everything else needs clarification and elaboration. The symbols need to be explained, better contrasting colours should be used for Pre-CC and Post-CC, etc. An intelligent reader perhaps is able to grasp everything in the present figure quickly, but the figures should be as easy-to-read as possible, not a test for the intelligence of the reader.}

\newpage
{\bf References:}
\bibliography{/Users/Lizzie/Documents/git/bibtex/LizzieMainMinimal}


\end{document}
