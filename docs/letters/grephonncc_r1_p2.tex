\documentclass[11pt]{article}
\usepackage[top=1.00in, bottom=1.0in, left=1.1in, right=1.1in]{geometry}
\usepackage{graphicx}
\usepackage{natbib}
\usepackage{amsmath}
\usepackage{lineno}
\usepackage{xr-hyper}
\externaldocument{..//grephonms}
\newcommand{\R}[1]{\label{#1}\linelabel{#1}}
\newcommand{\lr}[1]{line~\lineref{#1}}
\usepackage{hyperref}
\setlength\parindent{0pt}


\begin{document}
\setlength{\parindent}{0cm}
\setlength{\parskip}{7pt}

\bibliographystyle{/Users/Lizzie/Documents/EndnoteRelated/Bibtex/styles/besjournals}
\renewcommand{\refname}{\CHead{}}

Editor and reviewer comments (we provide below the full context of the three reviewers' comments) are in \emph{italics}, while our responses are in regular text. \\ 

{\bf Response to editors' comments:} \\  % See dothisGREPHON for notes on figures and overhaul of Section 3. 

We appreciate the editors' thoughtful comments and the chance to revise this work. ...

\begin{enumerate}
\item Completely overhauled the third section (`Building a new framework for growth × season length relationships') to be focused on the most critical questions to answer and clear suggestions of how to do it, \lr{startframework}-\lr{R3complaint1E}
\item Added a brief discussion of primary versus secondary growth on lines ...  and we return this as we develop our points  (concern of R1 and R3) 
\item New figure and clear discussion of importance of understanding growth rate (concern of R2) 
\item Added new figures ... 
\end{enumerate}

Overlapping concerns across reviewers:
\begin{enumerate}
\item Growth rate (err, maybe just R2) 
\item Different types of growth (primary versus secondary ... roots?):  R1 and R3
\item No clear framework! Not sure what to do next. R2 and R3 
\end{enumerate}

{\bf Reviewer comments:} 

{\bf Reviewer: 1}


\emph{In this manuscript, the authors address the critical question of whether the lengthening of the growing season directly translates into enhanced growth performances. The topic is highly relevant, especially given its implications for climate change predictions and carbon sequestration models. The authors cover a wide range of perspectives, from eco-physiological processes to the implications of predictive modeling, providing a comprehensive review of the current literature. Overall, this work is a thorough and well-organized examination of the complexities surrounding this topic, and the authors did excellent work in synthesizing diverse research findings. I believe the manuscript is suitable for publication, though I suggest that the authors further develop the following two key areas:}

We thank the reviewer for their positive feedback (especially for the breadth of perspectives we tried to cover) and suggestions for improvement. We have worked to address both these points as detailed below.\\

\emph{1) While the manuscript touches on the general concept of growth, it would benefit from a clearer differentiation between primary and secondary growth. Primary and secondary growth are processes with different temporal and environmental dependencies. In particular, primary growth is influenced by environmental conditions from previous growing seasons, while secondary growth is directly and exclusively responsive to current-year conditions. The manuscript discusses carbon allocation investments (e.g., on page 6, the competition between growth and reproduction). Still, it would be helpful to explicitly link these discussions to how primary and secondary growth might compete for resources differently. In this framework, I wonder if the hypothesis that a lengthened growing season might affect these two growth processes differently should be clearly raised, as there is a significant gap in the current understanding. Some recent studies suggest that extended growing seasons may enhance primary growth more than secondary growth due to the timing of carbon allocation and physiological constraints. This aspect deserves a deeper discussion, as it could affect future predictions of carbon storage.}

We completely agree and struggled with how much to address this in our initial submission, confining discussion of this complexity to the supplement, but we agree it needs a more obvious and direct discussion. To address this we have updated our Box text to discuss the complexity of growth and growing season length and added a new figure. We now cite this Box on \lr{R1box1}, \lr{R1box2} ...  XX and we also directly discuss this issue on ... XX \lr{R1growth} \\


\emph{2) The authors rightly mention that the definition of ``growing season" is debated, and this debate plays a critical role in understanding the relationship between season length and growth. To date, studies have varied significantly in their definitions, with some focusing on temperature thresholds while others emphasize the onset and cessation of physiological activity. This discrepancy complicates comparisons across studies. A more detailed explanation of the different definitions and their implications would greatly enhance manuscript accessibility, particularly for readers less familiar with the concept of phenology. Additionally, since the growing season length is a critical variable in models forecasting carbon sequestration, understanding its definition is crucial for interpreting results. I recommend a dedicated section that elaborates on the current definitions, the ongoing debate, and how different interpretations could lead to divergent conclusions about tree growth under climate change.}

We agree and have addressed this alongside more discussion of the definition of growth as these two issues are intertwined. Our new figure REFXX includes visualizations of several major definitions of growing season (XX and we add MORA figure?), which we discuss in the main text. We also have added an extended discussion of this to the supplement (given space limitations in the main text combined with the complexity of the issue) and we discuss this point more in our section on building a framework \lr{R1gslS}\lr{R1gslE}.\\

{\bf Reviewer: 2}


\emph{This study reviews the recent literature (36 papers) to group the changes in plant growth with season length, and finds that 58\% of studies supported the assumption of increased growth with longer season, while 36\% of studies did not. Then, the authors group the pathways or hypotheses that how climate change alters growing season length and then affects growth from a review of papers studying growth × growing season length. Finally, they outline how bridging these current divides while simultaneously integrating ecological theory could yield new advances in fundamental biology. The topic of ``why longer seasons with climate change may not increase tree growth'' is important and interesting. However, this review only groups some possible reasons for the relationship between growth and season length from a review of papers, and the underlying mechanisms remain unclear and the discussion also limit with future directions in this topic.}

We appreciate the reviewer finds this an important topic and understand their concerns that the underlying mechanisms remain unclear with limited future directions. Reviewer 3 also found the future directions in our manuscript needed work and in response we have completely re-written this section to address both reviewers' concerns. We detail the changes further below. \\


\emph{Major concerns:}\\
\emph{1. The possible reasons (pathways or hypotheses) that the authors grouped from the papers studying growth × growing season length are single, and have been well-know in previous study. However, the effects of diverse factors on plant growth are complex. This review does not give an overall framework for how changes in plant growth with the growth season via external drivers and internal constraints. For example, we don’t know how/whether climatic factors directly or indirectly alter growth seasons via internal constraints, and then affect growth. This is a key point to disengage and understanding when longer seasons will---or will not---lead to greater growth.}

We very much agree. As mentioned above we overhauled the last sections (effectively, the second half) of the manuscript to address. We now emphasize part of the reviewer's latter point as a critical question to address in `How do external drivers and internal constraints act together?' from \lr{R3complaint1S} to \lr{R3complaint1E}. We do group group papers by the individual hypotheses they test, but did not mean to indicate that these hypotheses are mutually exclusive and have updated the caption of Fig. \ref{fig:hypotheses} to clearly state this. We agree with the reviewer that the reality is far more complicated than any individual hypothesis here can express, and have added a clarifying statement in the text as well at \lr{R2complaint1nonmutexcl}. In revising our manuscript, we have increased the emphasis on what we need to know and future directions for research and modeling. We hope that by focusing more of the text on how very much we still need to resolve how climatic factors shape growth, we will encourage an increase in mechanism-centered experimental design and model development.\\

\emph{2. As shown in Fig. 1d-e in the manuscript, climatic factors alter plant’s physiological process (i.e., growth rate) rather than growth season length, and then dominate plant growth. This indicates that the physiological process plays a crucial role in regulating growth. In other words, if climate change reduces the growth rate, a longer growth season may induce higher, unchanged, or lower plant growth. In contrast, if climate change shortens the growth seasons, a higher growth rate may induce higher, unchanged, or lower plant growth. However, this review ignores the importance of growth rate, although some papers that they reviewed have mentioned the growth rate (see supp References: Colangelo et al., 2022; Zhang et al., 2021; Ren et al., 2019).}

We agree with the reviewer that growth rate is an important aspect of the interaction of total growth with climate and `season length,' and we did not mean to undervalue in our initial submission. We have retained text from our original submission discussing this (\lr{R2rate1S}-\lr{R2rate1E} and a figure) and extended the discussion of this in the latter half of the manuscript, see \lr{R2rate2S}-\lr{R2rate2E}.  In this expanded discussion of future directions, we have placed growth rate as an important target metric for increasing study resolution. In particular, \lr{R2rate3S}-\lr{R2rate3E} and a new figure (Fig. \ref{fig:phylomodel}) include explicit mention of the challenge of teasing out impacts of shifted rates from longer season as paramount to making progress. \\ %XXX FIGURES (MORA and the new phylo one).

% For simplicity, our review attempts to focus entirely on studies looking at growing season-length. While it is true that some of the paper we cite consider rate, their shared characteristic is linking phenology to total annual growth. 

\emph{3. As shown in Fig. 1g in the manuscript, this hypothesis indicates that the growing season length and growth are unchanged due to internal constraints. However, this hypothesis may be wrong and is not supported by the listed References, (e.g., Zohner et al., 2023 SCIENCE). 1) the growth onset is not unchanged, and should be advanced under global warming. For example, recent research has suggested that warming-induced earlier spring phenology is driven by increased photosynthetic carbon assimilation in the previous growing season (see Gu et al., 2022 NATURE COMMUNICATIONS). 2) The peak growth and their timing should be increased and advanced due to source-sink balances/limitations under global change (see Korner et al., 2015; Green et al., 2022 SCIENCE). 3) the growth end timing is also changed and is jointly controlled by climatic factors and internal constraints (see Zohner et al., 2023 SCIENCE).}

We are not advocating the hypotheses in Fig. \ref{fig:hypotheses}; indeed our review indicates that the impacts of growing season length on growth are likely dependent on species, progeny, location, method, and tree age, but seeing these impacts would likely take more data, collected in a way that makes it easier to see patterns across species and methods, which we now mention, see  \lr{R2complaint3S}- \lr{R2complaint3E}.\\ %None of them are correct for every situation, they simply reflect what is currently being tested experimentally. % Alana says: Okay, so 95\% sure the reviewer thinks we are advocating something here. Maybe the fig caption needs a statement of neutrality

\emph{4. The authors state that ``this research builds a mechanistic framework for when longer seasons will---or will not---lead to greater growth, with major forecasting implication''. However, after carefully reading this paper, I still struggle with how to examine and explore when longer seasons will---or will not---lead to greater growth. As described by above comment \#1, the mechanistic framework and the underlying mechanisms remain unclear.}

%Yeah, R3 said the same thing so we overhauled the whole damn end and you gotta admit -- we have some new ideas! But you might not like them ...
We appreciate the reviewer's concern and it echos some of the concerns of reviewer 3, thus we have completely redesigned and rewritten the `Building a new framework for growth $\times$ season length relationships' section of the manuscript. We now outline how understanding the complexity of this topic will require greater efforts to synthesize data across species and their diverse growth strategies while building a better macro-scale understanding of the relationship between season length and growth, better understanding the biological levels of constraints and their prevalence and integrating across such constrains and external drivers, see \lr{startframework}-\lr{R3complaint1E}. \\

\emph{5. For the suggestions for future work, the authors state that ```Using existing data and networks to partition levels of variation across drivers, New experiments to tease apart external \& internal drivers, Models that push forward theory and aid forecasting''. However, these suggestions are methods, and I still don't know exactly how/what to do next. For example, how to disentangle the changes in vegetation phenophases and season length by external drivers and internal constraints. This is important to improve our understanding of the relationship between season length and plant growth, and our prediction accuracy of the model. Therefore, this review should give what work have we done, and how/what work needs to be done next.}

We have worked to clearly outline next steps now (\lr{startframework}-\lr{R3complaint1E}) and have removed these sections and related figures. We are more clear in exactly what types of experiments we are suggesting (\lr{R3complaint5S1}-\lr{R2rate3E}) and how they should be integrated into larger models that better capture species-level differences in growth strategies and evolutionary history (e.g., \lr{startframework}-\lr{R3complaint5Ephylo}, \lr{R3complaint5S2}-\lr{R3complaint1E} and see Fig. \ref). We appreciate the reviewer pushing us to make the next steps we see as critical more clear and hope it will provide a valuable path forward.\\


{\bf Reviewer: 3}

\emph{1. LIMITED NOVELTY OF THE ECOLOGICAL DISCUSSION AND THE SUGGESTED FRAMEWORK}

\emph{The manuscript provides a nice and comprehensive discussion about the ecological phenomena underlying the potential relationship between growing season length and tree growth. That discussion then leads logically to the suggested mechanistic framework for subsequent studies. Both parts read well, but at the end I was wondering, how much really novel aspects were provided. When reading the text I got the feeling that yes, I have heard this before: we need to understand all of these ecological phenomena better, if we are supposed to understand and project the development and trees and forests under climatic change.}

We appreciate the reviewer's critique, especially their concerns that the framework and discussion did not feel novel enough. To address this (as discussed also above) we have completely overhauled the second half of the manuscript, re-writing the end of the drivers section (on species differences) and the entire  `Building a new framework for growth $\times$ season length relationships' section of the manuscript (\lr{startframework}-\lr{R3complaint1E}). We believe the revised manuscript provides an exciting and novel new framework to organize data, drive new observational studies and experiments and accelerate progress to predict when longer seasons lead to increased tree growth.\\

\emph{Similarly, in my view, the novelty of the suggested framework is limited. For instance, it has been known all too well for a long time that the mechanistic and the statistical models both have their strengths and weaknesses and that the two model categories should be better linked to each other. But how to do that? In other words, the discussion concerning the framework, too, reads well, but it lacks concrete novel aspects. One concrete suggestion is though made by recommending the use of the two major databases ITRDB and PEP725 simultaneously (Fig. 5). However, in my view it would be better to realize the idea in a separate study, not just to mention the idea here in a long list of to-do’s. And yes, integrative approaches over scales and study disciplines, and a better integration of ecological theory, are all definitely needed. But we have heard most, if not all, of that, too, already before.}

% Noted! And R2 complained also, so we overhauled the last third and streamlined the second section. Hopefully you're less bored now?\\
We can see the reasons for these concerns, and they echo some of Reviewer 2's. Certainly we do not want to provide a long list of to-do's, but a framework that could help us best understand current data and patterns and drive the design and utility of new work, and we have revised the manuscript towards this aim. We have streamlined and shortened the section on `Controllers on growth $\times$ season length relationships' while still trying to address the complexity of the problem and results from studies to date. Further, as previously mentioned the  `Building a new framework for growth $\times$ season length relationships' section is completely re-written with a clear framework to build on and three central questions to address, as well as specific suggestions of how to address them. Finally, we have removed several of the figures or moved to the Supplement (including the previous Fig. 5 the reviewer mentions) and added new figures that better stress the framework we propose and its value. \\

\emph{In brief: to my understanding the suggested framework is a to-do list which is not especially novel. However, the authors may disagree with me about this. If they do, then I recommend that they point out the truly novel aspects more clearly. Now the possible novel aspects are hidden in the textbook- like text.}

%We think it's less boring now!\\
We appreciate the reviewer pushing us for more here! We believe there are novel aspects in our submission, but they are much clarified now in the `Building a new framework for growth $\times$ season length relationships' section and hope the reviewer will agree. \\

\emph{Even in its present form the suggested framework would fulfil the requirements of most journals. However, I doubt if the framework fulfils the rigorous requirements of NCC.}

We believe our efforts to integrate across diverse fields studying the relationships between growth and season length is an important contribution and that the current manuscript is much improved in terms of highlighting the novelty of our framework and its value, and we thank the reviewer for their help in improving the manuscript.\\

\emph{2. DIFFERENT ASPECTS OF GROWTH SHOULS BE DISCUSSED}

\emph{Unless I am mistaken, photosynthesis is taken in the study as one metric of growth. I can see the motivation for that, but because that is contradictory to the established use of these concepts in plant ecophysiology and ecology, this issue should be elaborated. A failure to do that would not be a good start for the integrative approaches the authors are calling for.}

We completely agree and this was an important concern also raised by Reviewer 1. As we mentioned in response to Reviewer 1, we struggled with how much to address this in our initial submission, confining discussion of this complexity to the supplement, but we agree it needs a more direct discussion. To address this we have updated our Box text to discuss the complexity of growth and growing season length and added a new figure. We now cite this Box on \lr{R1box0}, \lr{R1box1}, \lr{R1box2} ...  XX and we also directly discuss this issue on ... XX \lr{R1growth}. \\

\emph{Along the same line of reasoning, different aspect of growth as such should be better elaborated and discussed. The discussion on determinate species on page 6, for instance, is relevant only for height growth, not for diameter growth, or accumulation of biomass. Because of these reasons, a general theory between growing season length and `growth' maybe even impossible, not only difficult to achieve. In my view, this overall limitation should be addressed more clearly in the study. What exactly is aimed at in this and the related studies? If the aim is to examine carbon sequestration, then the discussion on determinate species is not relevant. Those species fix carbon long after their height growth cessation has taken place.}

These are all good points that we agree with. Our aim is to work across the definitions of growth, which we have tried to clarify in this submission, see \lr{R1growth}. We also have revised the section that mentioned determinate and indeterminate growth into a broader and more relevant discussion of the problem (\lr{R3complaint2S}-\lr{R3complaint2E}). Further, we now discuss the major problem of species differences in growth strategies and how to tackle it in our revised `Building a new framework for growth $\times$ season length relationships' section, especially at the beginning, see \lr{startframework}- \lr{R3complaint5Ephylo}.\\

\emph{3. UNCLEAR DOCUMENTATION OF METHODS}

\emph{The manuscript lacks a Method-section, so the methods are documented only in the supplementary material. Whether that is allowed or not in NCC is up to the editors, but more seriously, the documentation of the methods needs to be edited for better clarity. Most importantly, the method documentation should tell as clearly as possible the rules, how a given examined published paper found its place in Fig. 1. I read the method documentation several times, but this crucial issue remained largely unclear to me. I understand that given the diversity inherent in the original studies examined, documentation of the rules for the classifications done in the present study is not an easy task. However, methods need to be properly documented also when the documentation is not easy.}

We appreciate this concern as making our approach clear and our methods reproducible is a main aim of all our work. To address this (while trying to meet NCC requirements for length and this format), we have included a section in \emph{Literature review methods} of the Supplement to better document our methods for assessing hypothesis testing across the review. Additionally, we have prepared all the data from this review and related code for the figures and will publish it upon manuscript acceptance. \\

\emph{4. MISCELLANEOUS COMMENTS}

\emph{Figures and their legends should be as clear and as self-explanatory as possible. Regarding this, the crucial Fig. 1 and its legend needs a major editing. The two panels on the right are clear and informative, but almost everything else needs clarification and elaboration. The symbols need to be explained, better contrasting colours should be used for Pre-CC and Post-CC, etc. An intelligent reader perhaps is able to grasp everything in the present figure quickly, but the figures should be as easy-to-read as possible, not a test for the intelligence of the reader.}

We apologize for this and have worked to make our figures, figure legends and captions clearer in this version, especially Figure 1. We were also advised by NCC that---should the paper be accepted---their style guidelines would be specifically applied, which may include additional color and legend changes. \\

\newpage
{\bf References:}
\bibliography{/Users/Lizzie/Documents/git/bibtex/LizzieMainMinimal}


\end{document}
