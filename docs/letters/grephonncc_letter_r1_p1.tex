\documentclass[11pt,a4paper]{article}
\usepackage[top=1.00in, bottom=1.0in, left=1in, right=1in]{geometry}
\usepackage{graphicx}
\usepackage{sectsty,setspace,natbib,wasysym} 

\begin{document}
\bibliographystyle{/Users/Lizzie/Documents/EndnoteRelated/Bibtex/styles/naturemag}
\begin{figure}[htbp]
\hspace*{14cm}                                                           
\hspace{-35ex} \includegraphics[width=0.5\textwidth]{/Users/Lizzie/Documents/Professional/images/letterhead/ubc/Faculty of forestry.png}
\end{figure}
\vspace{-10ex}
\begin{small}
\noindent 2424 Main Mall \\
\noindent Vancouver, BC Canada V6T 1Z4\\
\noindent Ph: 604.827.5246\\
\end{small}
\vspace{2ex}\\
\pagenumbering{gobble}

\noindent Dear Dr. Armarego-Marriott:
\vspace{1.5ex}\\
Please consider our revised manuscript, ``Why longer seasons with climate change may not increase tree growth,'' (NCLIM-24061674) for consideration as a Perspective in \emph{Nature Climate Change}. 
\vspace{1.5ex}\\
The idea that longer seasons lead to increased plant growth is an intuitive tenet across multiple fields of biology, and a critical assumption of most global climate models \citep{friedlingstein2022global}. A suite of recent studies, however, have challenged this assumption \citep[e.g.][]{dow2022warm,green2022limits}, increasing concerns that future climate change impacts could be underestimated \citep{green2022limits,korner2023four}. 
\vspace{1.5ex}\\
To address this growing debate, we present a Perspective that reviews recent literature to understand the underlying biological mechanisms that may limit how trees grow as climate change extends seasons, then provides a path forward.  We show how increased cross-disciplinary efforts could build a universal model that can predict when, where and how climate change may increase tree growth. 
\vspace{1.5ex}\\
Comments from three reviewers and yourself have led us to completely overhaul our manuscript. We streamlined parts of the first half of the manuscript so we could extend our second half, which now clearly lays out a novel and important framework to organize results and guide future experiments. New figures and text highlight the important role of community ecology theory and evolutionary history---two areas that are effectively missing from current research---in developing models, predictions and designing studies. Building on this, we next review the three most critical questions to address for progress. We believe our revised manuscript is far more exciting and provides a novel and motivating path forward for the field, and we detail these changes in our point-by-point response below. 
\vspace{1.5ex}\\
This Perspective benefits from an interdisciplinary authorship team, leveraging expertise from dendrochonology, community ecology, physiology and phylogenetics. We believe this perspective could therefore reshape research into tree growth with climate change and we hope that you will find it suitable for publication in \emph{Nature Climate Change}. This manuscript is not under consideration elsewhere, and all authors approved of this version for submission. 
\vspace{1.5ex}\\
Sincerely,\\

\includegraphics[scale=1]{/Users/Lizzie/Documents/Professional/Vitas/Signatures/SignatureLizzieSm.png} \\

\noindent Elizabeth M Wolkovich\\
Associate Professor of Forest \& Conservation Sciences\\ 


\newpage
\bibliography{..//..//bibtex/grephonbib.bib}
\end{document}
