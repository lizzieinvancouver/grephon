\documentclass[11pt,a4paper]{article}
\usepackage[top=1.00in, bottom=1.0in, left=1in, right=1in]{geometry}
\usepackage{graphicx}
\usepackage{sectsty,setspace,natbib,wasysym} 
\usepackage{lineno}
\usepackage{xr-hyper}
\externaldocument{..//grephonms}
\newcommand{\R}[1]{\label{#1}\linelabel{#1}}
\newcommand{\lr}[1]{line~\lineref{#1}}
\usepackage{hyperref}

\begin{document}
\bibliographystyle{/Users/Lizzie/Documents/EndnoteRelated/Bibtex/styles/naturemag}
\begin{figure}[htbp]
\hspace*{14cm}                                                           
\hspace{-35ex} \includegraphics[width=0.5\textwidth]{/Users/Lizzie/Documents/Professional/images/letterhead/ubc/Faculty of forestry.png}
\end{figure}
\vspace{-10ex}
\begin{small}
\noindent 2424 Main Mall \\
\noindent Vancouver, BC Canada V6T 1Z4\\
\noindent Ph: 604.827.5246\\
\end{small}
\vspace{2ex}\\
\pagenumbering{gobble}

\noindent Dear Dr. Armarego-Marriott:
\vspace{1.5ex}\\
Please consider our revised manuscript, ``Why longer seasons with climate change may not increase tree growth,'' (NCLIM-24061674A) for consideration as a Progress Article in \emph{Nature Climate Change}. 
\vspace{1.5ex}\\
The idea that longer seasons lead to increased plant growth is an intuitive tenet across multiple fields of biology, and a critical assumption of most global climate models \citep{friedlingstein2022global}. A suite of recent studies, however, have challenged this assumption \citep[e.g.][]{dow2022warm,green2022limits}, increasing concerns that future climate change impacts could be underestimated \citep{green2022limits,korner2023four}. 
\vspace{1.5ex}\\
To address this growing debate, we present a Progress Article that draws on recent literature combined with new perspectives to help advance the field beyond its current debate. We argue that the field needs to better understand the underlying biological mechanisms that may limit how trees grow as climate change extends seasons to advance and we provide a path towards this.  We show how increased cross-disciplinary efforts could build a universal model that can predict when, where and how climate change may increase tree growth. 
\vspace{1.5ex}\\
Comments from one reviewer and yourself have led to improvements throughout the manuscript, from the abstract to the end. We detail these changes in our point-by-point responses. As requested, we have provided a set of point-by-point responses to your comments in this letter (below) and a separate set of point-by-point responses to the reviewer. % Add that the DIFF file does not have complete referencing but shows changes?
\vspace{1.5ex}\\
This Progress Article benefits from an interdisciplinary authorship team, leveraging expertise from dendrochonology, community ecology, physiology and phylogenetics. We believe this article could therefore reshape research into tree growth with climate change and we hope that you will find it suitable for publication in \emph{Nature Climate Change}. The current manuscript is 3874 words long with a box of 540 words, 4 figures (plus one included in the box) and 140 references (we certainly could reduce the reference number if needed). This manuscript is not under consideration elsewhere, and all authors approved of this version for submission. 
\vspace{1.5ex}\\
Sincerely,\\

\includegraphics[scale=1]{/Users/Lizzie/Documents/Professional/Vitas/Signatures/SignatureLizzieSm.png} \\

\noindent Elizabeth M Wolkovich\\
Associate Professor of Forest \& Conservation Sciences\\ 

\newpage
\noindent {\bf Response to editor's comments:}  \\

\noindent \emph{Firstly, we want to discuss whether it would be possible to change the format from Perspective to Progress Article (PA). PAs are effectively very similar to Rev/Pers from a format point of view, but we have used them when a field is still at an earlier stage, as a signal the readers that this area is growing- we hope this may in some ways may help with the reviewer’s feeling that a Review on the topic is premature.}\\

\noindent We thank the editor for this suggestion as we feel it well captures the state of the field. Further, we agree that some of the concerns by reviewers may be related to the type of article overstating our aims in this manuscript---which is to help advance research on tree growth and extended seasons through outlining major gaps, which cannot easily be addressed currently without new data, experiments and perspectives. To that end we have edited the manuscript throughout to try to clarify this (e.g., \lr{fornccS}-\lr{fornccE}, \lr{notallmech}-\lr{morebox1}).\\

\noindent \emph{Secondly, we would ask if it would be possible to further develop the possible mechanisms, in accordance with the concerns below:
‘outlining all possible mechanisms that could affect these relationships is important for readers who wish to test them in future research. Unfortunately, this part of the content is currently not complete coverage.’}\\

\noindent We have worked to address this, please see our response to Reviewer 2.\\ 

\noindent \emph{I’m also wondering if it’s possible to show something that looks more like a mechanistic framework as requested by the reviewer, albeit one with hovering question marks -- but I also realise that this might look quite similar to Fig1. Would it be possible to redesign this slightly? Could other factors be included in this to extend in the way Reviewer 2 suggest (incl. developmental processes?)}\\

\noindent We spent a while trying to integrate this into Figure 1, but we feel the current mechanisms and developmental processes are not well enough known to layer on. As an alternative way to address this, we have extensively altered the text in the box (especially \lr{forbigKref2S}-\lr{forbigKref2E} and \lr{forbigKNUTSetcS}-\lr{forbigKNUTSetcE}) and altered the figure in the box to include more of these processes. We agree that these are critical areas of research for progress in this area, which we now address in the main text (see responses to Reviewer 2) and we also have integrated and referenced the box where we develop these areas further throughout the main text (e.g., \lr{forbigKnomodel1}, \lr{morebox}, \lr{morebox1}). \\

\noindent \emph{As well as the broader responses to R2, I’d also like to make some formatting suggestions to align with our style at this stage. If you’re able to send through a MWord version of the text, I could provide some mark-up to help make these changes throughout.}\\

\noindent We have worked to address these concerns, including overhauling the abstract, improving the text throughout, especially in the section on paths forward, where we have added a new subheader (`Integrate phylogeny and traits to guide research') and shortened this new section, as well as improved all other sections in this part and re-ordered them. We have also shortened header titles throughout, changed the formatting of references, and other requested edits. If any appear missing, please let us know.\\

\newpage
\bibliography{..//..//bibtex/grephonbib.bib}
\end{document}
