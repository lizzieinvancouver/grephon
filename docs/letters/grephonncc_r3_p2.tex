\documentclass[11pt]{article}
\usepackage[top=1.00in, bottom=1.0in, left=1.1in, right=1.1in]{geometry}
\usepackage{graphicx}
\usepackage{natbib}
\usepackage{amsmath}
\usepackage{lineno}
\usepackage{xr-hyper}
\externaldocument{..//grephonms}
\newcommand{\R}[1]{\label{#1}\linelabel{#1}}
\newcommand{\lr}[1]{line~\lineref{#1}}
\usepackage{hyperref}
\setlength\parindent{0pt}


\begin{document}
\setlength{\parindent}{0cm}
\setlength{\parskip}{7pt}

\bibliographystyle{naturemag}
\lrenewcommand{\lrefname}{\CHead{}}

Reviewer comments (we provide below the full context of the editor's and reviewer's comments) are in \emph{italics}, while our responses are in regular text. \\ 

{\bf Editor comments:} \\

\emph{Please do consider the final very minor comment from Reviewer 3 regarding wording.}\\

Done. See below reply.\\

\emph{Please do remove the date from the cover page.}\\

Done. \\

\emph{Please move the author contributions to the bottom of the article (with acknowlegements).}\\

Done.

\emph{Is it possible to consider a title that highlights the current conflict in studies. Something like `Reconciling impact of longer seasons with climate change on tree growth'?}\\

We have altered our title slightly (another alternative seems too long: Why longer seasons with climate may not lead to the increased tree growth that forecasts assume). We are not comfortable with 'reconciling' as it does not seem what the paper does, and other alternatives suggested by co-authors were not well received by most co-authors (who generally like the current title).  \\ 

\emph{Overall, we'd ask if it's possible to try to shorten and remove the questions from the headings (which would be a bit more in line with our style).}\\

We edited all the flagged subheaders and all subheaders should now meet the character length requirements and lack punctuation. We also edited the text to meet these requirements so we no longer mention addressing major questions in the field.\\

\emph{It may help to add 1-2 more sentences at the top of the piece explaining a bit more about why we care so much about this question in broad terms (for the climate change community). More importantly, could you also consider putting in a short conclusion paragraph, that again brings back in the importance of understanding this and the implication for CC/carbon budgeting?}\\

Done. In the introduction we now start with:

\begin{quote}
How plant growth shifts with warming has ramifications for the stability of ecosystems and the global climate system itself \citep{ipcc2021}. Because plants act as one of the greatest potential stores for carbon emissions how and how much their growth changes with climate change is one of the top predictors of future climate change \citep{ipcc2021,friedlingstein2022global}. 
\end{quote}

In the conclusions, we now end with: 

\begin{quote}
Starting now to leverage data across species to inform and design new large-scale studies and experiments will help build accurate models of future forest and related carbon dynamics, with implications for projections of carbon sequestration, carbon markets and climate stability. Emissions reduction and mitigation strategies today depend on growth and yield models built on assumptions that---if incorrect for certain species, regions or climate change scenarios---could endanger the success of current efforts \citep{ellis2024principles}. Thus, we argue that starting now to gather better data, build better models to improve understanding of how climate change has and will impact tree growth is critical to durable and resilient policies to limit and adapt to future climate change.
\end{quote}

\emph{We do want to try to keep this short, as close to 3500 words as possible. This is not an incredibly strict limit, but as you do a final read, please do consider any streamlining options.}\\

We have cut several sentences and a paragraph from the Box text (now 400 words). If there are other streamlining suggestions, please let us know.\\

\emph{Our style limits use of certain terms, including novel/new. Please do edit "new" - line 200, 209 and check throughout.}\\

We removed five usages of `new,' including those flagged and one use of the word `novel.'\\ 


\emph{Similarly, we ask that 'we/our' be used only for author teams not for society (to avoid confusion'. Please check throughout.}\\

Done, we removed three usages of `we' and two usages of `our' to meet these style guidelines.\\

\emph{edit out ``ramifications for how much growth is expected to shift with warming".}\\

We assume the suggestion was to remove the entire sentence and have done that (`If true, this would have major ramifications for how much we expect growth to shift with warming.).\\

\emph{ln208, suggest ``though currently data is only available for a few species"}\\

Done. \\

\emph{Please ensure that all Figure and Table legends begin with a short title sentence that does not reference any of the panels of the section, and does not run on into a longer sentence (this is currently an issue mostly for figures in Supps). For example, Fig2 is currently more of a result than a title, as are Fig3/4.\\
- Overall, please do review the legends to ensure that they are fully descriptive of the figures. We need to ensure that display items 'stand alone'. This means that it will be necessary to include more information, including descriptions of all of the abbreviations, categories and symbols used within the figure. This will need to be updated particularly for figure 1,}\\

All captions have been updated per these requirements. \\

\emph{We note that the ORCID for authors Alana Chin, Catherine Chamberlain, Frederik Baumgarten, Kavya Pradhan, Rubén D. Manzanedo and Janneke Hille Ris Lambers are not linked to this manuscript}\\

All of the co-authors have confirmed to me (EM Wolkovich) that they have done this, though some cautioned they had multiple profiles (Hille Ris Lambers) or do not see the manuscript linked to their profile (Manzanedo) or already had an ORCID attached to their profile (Chamberlain). I hope \emph{Nature Climate Change} can help assist to make sure all the ORCID are connected.\\

\emph{Please ensure that all Figure and Table legends begin with a short title sentence that does not reference any of the panels of the section, and does not run on into a longer sentence (this is currently an issue mostly for figures in Supps). For example, Fig2 is currently more of a result than a title, as are Fig3/4.\\
- Overall, please do review the legends to ensure that they are fully descriptive of the figures. We need to ensure that display items 'stand alone'. This means that it will be necessary to include more information, including descriptions of all of the abbreviations, categories and symbols used within the figure. This will need to be updated particularly for figure 1,}\\

All captions have been updated per these requirements. \\

\emph{Please include a COI statement.}\\

Done. \\

\emph{Supp Figs/Tables: Please do ensure that the figures/table caption follow the same rules for the main text- mainly that they start with a figure legend that summaries the main point of the display item.}\\

Done.\\

\emph{Review type articles allow the authors to highlight key papers in the reference list. Please add descriptive text to top papers in
your reference list (this should be limited to no more than 12 references only) if you wish. See http://rdcu.be/iMP5 and http://rdcu.be/iMP6 for examples.}\\

We now highlight several references (seven, I believe). Noted with \emph{Highlighted ref:} in the bibliography now. \\ 

{\bf Reviewer: 2}

\emph{The revised version of the manuscript is now clear. I have no further comments or suggestions for improvement.}\\

{\bf Reviewer: 3}

\emph{This is a revision of a manuscript I reviewed in October 2024. I was invited to review a revised version already in early 2025, but at that time I was not able to do that within the schedule provided by the editors. So, unless I am not mistaken, this is a second revision of the manuscript. I did not see the first revision, nor did I see the reviewer comments of the other reviewers on the original version. This time I read the comments of Reviewer 2, which I guess were given for the first revision I did not see.\\
My main concern with the original manuscript was the lack of novelty, or lack of clarity about what is the novelty, warranting the publishing of the paper in NCC. Now I am glad to say that concerning this critical aspect, the manuscript has improved a lot. Especially the role of the phylogenetic analyses is now clearer and more concrete. Having said that, I am not sure this time, either, if the study will provide such a groundbreaking new framework to the ecological community as what the authors may have in mind. Here I refer also to the limitations mentioned by Reviewer 2 in his comments, obviously given on the first revision. However, even in the top journals it would not be fair to require any final frameworks solving all of the problems, because that would be beyond the capabilities of any scientific research. This is so especially in the discipline of ecology where it is impossible to develop such highly generalised approaches and Laws of Nature which are the mainline in physics.\\
What the authors may consider is to use a slightly more modest wording, in the spirit that they provide some insight for making progress in this field of science. Following that line of reasoning, for instance, I would not use the concept ‘mechanistic model’ in this context because it can be understood to mean a model covering all of the things we should take into account. However, I leave that to the authors.}\\

We have removed the phrase `mechanistic model' from the revised text and now only use the word `mechanistic' twice. \\


\emph{I was happy to notice that the authors have sufficiently addressed the smaller concerns I had on the original submission.\\
I guess the above comments are in line with what the editor stated in her email message about the change of the format from a Perspective to a Progress Article. So, in all, I am happy to recommend that the manuscript reviewed is accepted for publication as a Progress Article in Nature Climate Change.}\\


\bibliography{..//..//bibtex/grephonbib.bib}

\end{document}


