% Stuff I extracted during revision for NCC in early 2025
% Old word counts: about 4K plus 730 in boxes (4,550 words currently and by Jan 2024: Up to 5K ... feels a little too long to me, down to 4840 on 21 Feb 2024 ... down to 4K on 9 March ... but I assume that did not include boxes) on 24 June 2024: 3764 K plus 176 in one box (I got rid of a box). 

\subsection*{Species-level variation} 
% To date, a handful of studies have mentioned species differences (Fig. \ref{fig:hypotheses}) but almost none made or tested predictions on how species differ based on existing theory. 
The effects of these external and internal drivers are likely to vary across species, a reality rarely acknowledged by most studies (Fig. \ref{fig:hypotheses}c). Species identity, however, strongly predicts variation in growth $\times$ season length relationships \citep[e.g.][]{cuny2012life,michelot2012comparing} and thus is likely critical for understanding the widespread observed variation. Biogeographical patterns in climate and assembly within communities also predict species should evolve towards different optima and different strategies \citep{Ackerly:2009ly,buckley2012functional}. Leaf strategies (e.g. leaf mass per area, longevity) vary strongly between evergreen and deciduous species, but also within each group---where variation in `determinacy' defines the timing and investment of shoot growth and leaf emergence. Determinate species have most of their leaf material prebuilt in overwintering buds, generally unfolding their entire canopy within a few weeks each season, while indeterminate species continue to produce new shoots including leaves over the growing season \citep{kikuzawa1982leaf,Lechowicz:1984cr}. Such differences would influence the extent to which the growth of different species respond to increases in growing season length, even under identical conditions. Current studies span a wide range of species (we found  \Sexpr{sppnum} species from \Sexpr{gennum} genera across \Sexpr{papernum} papers), making the aim of identifying a common relationship between growth and growing season length with current studies especially difficult.

Phylogenetic ecology provides tools to study imprints of past selection, which often shape species-level differences today---limiting how well species are adapted to current conditions and potentially constraining their responses to rapidly changing conditions \citep{Ackerly:2009ly}.  %Many species show evidence of previous selection, seen when evolutionary relationships (usually represented through phylogeny) predict plant responses and lead to clade-level similarities. 
Most studies testing for such effects of evolutionary history on plant responses find them \citep[e.g.][]{phenophylo}. This includes new work on physiological traits \citep{avila2023evidence}, and previous physiological syntheses finding results suggestive of strong phylogenetic relationships \citep[e.g.][]{way2010differential}.

% Some bits of this were moved down to existing datasets section (May 2024). 
% Studies could leverage community and phylogenetic ecology theory to make useful predictions for when and where growth $\times$ growing season should be most apparent. Community ecology predicts trade-offs along an acquisitive to conservative axis, where some species grow rapidly and more flexibly to take advantage of resources, but are less defended against herbivores and compete poorly at low resource levels, whereas other species compete well at low resource levels, but at the expense of growing slower and conservatively \citep[][]{Grime:1977sw}. These ideas would predict indeterminate acquisitive species, such as poplar, to grow more with longer seasons, while conservative species, such as beech, may not. Functional traits could further refine these predictions, with where species fall along the acquisitive versus conservative trade-off defined by suite of leaf, wood and reproductive traits \citep[][]{diaz2016}. Under this framework, species with low leaf mass per area, diffuse vessels and consistent investment in fruit would show stronger shifts in growth with changing growing season length---assuming no other factors (e.g. drought or high temperatures) become limiting.


\section*{Building a new framework for growth $\times$ season length relationships} 
% "Crossdisciplinary: viewing one discipline from the perspective of another. Multidisciplinary: people from different disciplines working together, each drawing on their disciplinary knowledge. Interdisciplinary: integrating knowledge and methods from different disciplines, using a real synthesis of approaches."
 % Building a mechanistic framework of when and where...  % Building a cross-disciplinary multi-level ...
Predicting when and where longer seasons lead to increased growth may seem overwhelming given the diversity of potential drivers and complexity of species-level differences, but together they offer a set of testable hypotheses that could rapidly advance progress---if tackled with a more cross-disciplinary approach. Such changes may take time, but major hypotheses can be tractably tested now. 

Taking advantage of existing data sets and ongoing experiments could provide tests of variation in growth at relevant organizing levels---individuals, populations, species and ecosystems---to provide a benchmark when comparing the effect sizes of external drivers of variation (e.g. climate, pest outbreaks). While multiple papers report a lack of relationship between growth and growing season length, we have no fundamental understanding of what the effect size of this relationship should be, and thus no way to know if we have good power in current studies to detect it. New experiments could also help address this gap and could be designed to directly compare effects of external versus internal drivers on growth. Combining observational and experimental data in models that build up from internal limits to external drivers and include species-level variation would then provide predictions while helping to refine theory. More tractable changes within fields would also help---the high variation in observed growth responses to longer seasons (Figs. \ref{fig:heatmaps}, \ref{fig:sppfinds}) could be partly reduced through standardized measurements (see Box: Standardized measurements) and a broadening of perspective within fields (see `Extending disciplinary focus' in Supplement). 
% This approach requires building fundamental biological knowledge in a suite of areas across physiology, dendrochronology, life history, ecology and evolutionary biology. These suggestions thus apply to understanding this relationship at the individual (organismal) level, though they make predictions at larger (e.g. ecosystem) scales, and are highly applicable to ecosystems dominated by one species \citep[e.g.][]{chen1999effects}. % These suggestions apply to those wanting to specifically address hypotheses about thevau l relationship between growing season length, phenology and growth at the individual (organism) level


\subsection*{Using existing data and networks to partition levels of variation across drivers}
% Critical bits of this were moved up (May 2024). 
% Predicting when longer seasons increase tree growth requires understanding the scale of growth variation at relevant organizing levels---individuals, populations, species, to provide a benchmark when comparing the effect sizes of external drivers of variation (e.g. climate, pest outbreaks). While multiple papers report a lack of relationship between growth and growing season length, we have no fundamental understanding of what the effect size of this relationship should be, and thus no way to know if we have good power in current studies to detect it. Estimates of how growth shifts with elevation (Fig. \ref{fig:gxelev}) likely include responses from both plasticity (within-individual variation) and local adaptation (population-level variation) and thus could be an upper bound on our expectations, yet elevational trends to date appear relatively weak and noisy---suggesting this is only part of our missing mechanistic understanding. However, a suite of current experiments, observational networks and existing databases could address this gap. 

Taking advantage of existing ecological and field global change experiments could help bridge across the two major fields currently studying growth $\times$ season length relationships---physiology and dendrochronology---and their contrasting timescales. We found most physiological studies of growth $\times$ growing season length relationships studied 1-2 years of dynamics, usually of juvenile trees, while tree ring studies focused on synthesizing across decades of adult tree growth. Perhaps because of this dichotomy, tree ring studies often study lag effects, while they are rarely mentioned in physiological studies, but current large-scale experiments on heat \citep[e.g. SPRUCE,][]{hanson2017attaining}, moisture via drought or irrigation \citep[e.g. DroughtNet, Pfynwald,][]{smith2016drought} and other factors (e.g. $\text{CO}_2$ in FACE) have increasingly been used to test ecological `memory' \citep[e.g. ][]{flinker2021promise, schweiger2022transgenerational}. They thus could help scale up from smaller and shorter-time scales of physiological studies, potentially to ecosystem-level dynamics, such as carbon cycling \citep{ding2021plant,jensen2019simulated}. Building on available data and infrastructure could also bridge this gap, for example, adding dendrometers to provenance studies \citep[or other ways to measure growth, e.g.,][]{montwe2016drought} and locations with established phenological sampling and vice versa. Such efforts may be especially valuable in sites across elevational and latitudinal gradients (e.g. PSP, Feeley elevation network, Forest Inventory and Analysis). These sites in turn could be priority locations for xylogenesis and focused physiological studies. 

% Existing common garden studies on trees (often called `provenance trials' in forestry) provide an opportunity for more robust tests of population and individual variation. Given that many common garden studies have data on phenology \citep{aitken2016} and are designed to tease out population versus inter-annual variation, collecting tree ring data from them seems a rapid way to estimate variation across these two levels. New measurements of biomass or greenness within a growing season could also help compare support for whether internal limits are universal at solstice \citep{zohner2023effect}, variable by population \citep{soolanayakanahally2013timing}, or some mix.  % Given how old some common gardens are, research may also be able to examine impacts of biotic and abiotic disturbances or effects of climatic variation. 
% Common gardens not collecting regular phenology, or annual growth data, could start. 
% Common garden studies are designed to tease out population versus inter-annual variation, but the variation across these levels in growth $\times$ growing season length relationships is rarely studied. 

Existing open data repositories could test predictions from community ecology for species-level variation in responses to external drivers. Combining large-scale databases of tree rings and vegetative phenology (e.g. the International Tree Ring Database, ITRDB, and the Pan European Phenology project, PEP725, see Fig. \ref{fig:itrbdpep}) would provide a major spatially and temporally diverse dataset to compare how external climatic drivers, species and population together explain growth $\times$ season length relationships. While the low spatial and taxonomic overlap between these databases currently poses challenges, combining these datasets with theory from community ecology may also allow us to identify which species will grow more with longer growing seasons. Community ecology predicts trade-offs along an acquisitive to conservative axis, where some species grow rapidly and more flexibly to take advantage of resources, but are less defended against herbivores and compete poorly at low resource levels, whereas other species compete well at low resource levels, but at the expense of growing slower and conservatively \citep[][]{Grime:1977sw}. Integrating these perspectives within a broader framework would provide predictions that researchers could test from combined datasets, specifically that longer growing seasons will increase growth for species with regular reproduction \citep[no masting, see also new masting database in][]{hacket2022mastree+}, an acquisitive strategy, from clades that are historically (on an evolutionary timescale) plastic, in locations that are warm---but not too warm---and moist. 


\subsection*{New experiments to tease apart external \& internal drivers}

Given the complex effects of external drivers and internal constraints on growth $\times$ season length relationships, fully disentangling them will likely require new experiments. Changes in growing season length covary with other environmental changes, in particular longer seasons are usually warmer seasons \citep{ipcc2021}. Thus, experiments to robustly tease these drivers apart seem a paramount need, especially across multiple species spanning diverse strategies. Similarly, factorial experiments that manipulate season length (via early growth or delayed senescence), while additionally manipulating external abiotic (e.g. heat waves, droughts) and/or biotic (e.g. pests, competition) drivers could allow us to compare the effects of these drivers on tree growth. Such experiments could also test for lag effects, if sampled multiple years after the manipulations (versus the common practice of destructive sampling at the end of the treatment growing season). While such experiments are most easily done for juvenile trees, they could also be done on adult trees, given investment in infrastructure. % Similarly, experiments to compare impacts of external biotic and abiotic drivers are critical---if carefully designed they would also provide insights on potential constraints.  

Efforts to design and launch large-scale experiments should start now. Long-term experiments on adult trees that manipulate temperature, precipitation and season length could test a suite of drivers at relevant lifestages. Such experiments could robustly compare drivers and become a resource for testing the underlying mechanisms for constraints, if properly measured and designed. This would mean careful measurements of carbon allocation, including to reproductive output, and tissue lost to frost and biotic drivers, and choosing species to maximize divergent strategies and provide the potential for genomic and related studies (e.g. \emph{Populus, Quercus}). Given the potential role of evolutionary history, selecting for these varying strategies within a clade, or---if not feasible---correcting for phylogenetic distance would provide more robust tests of how strategies influence the growth $\times$ season length relationship. % Given an increasing number of studies across more species, a careful synthesis of studies across species could further test for the role of evolutionary history. 
These highly measured experiments  would represent a major investment to tackle this question in one location, and could form part of a broader network of sites to test these relationships at larger spatial scales. Distributed experiments to measure growth and phenology (ideally wood and vegetative) of multiple provenances of multiple species across sites could estimate variation---and potential constraints---that operate at different organizing levels. 

\subsection*{Models that push forward theory and aid forecasting}
Efforts to bridge observational trends with experimental insights will need statistical and more mechanistic models that can bridge across temporal and organismal scales while testing the major hypotheses. New statistical models should include the separate effects of temperature, moisture and growing season length while partitioning individual, population and species-level variation---thereby providing broad-scale estimates of the effects of the major external drivers versus potential internal constraints (which may be apparent as within-season and/or population differences). Including species-level effects while also integrating phylogenetic relationships between species could then test for the role of evolutionary history in shaping responses, while adding in site $\times$ year-level effects of biotic disturbances could begin to compare across abiotic and biotic external drivers. Such models should be built alongside a suite of mechanistic process-focused models that scale up. For example, one model could build from carbohydrate balance and cell division \citep[e.g.][]{locosselli2017dendrobiochemistry} to predict growth dynamics observed in xylogenesis, while another could build from phenology, including frost disturbance and reproduction, to predict growth for different species \citep[e.g.][]{chuineJTB}. 

The success of modeling approaches will likely depend on how nimbly they respond to new findings and how well they make predictions for new studies to test, which likely requires bridging across statistical and mechanistic models. As new experiments identify potential internal growth constraints and what level they operate on (universal, population or otherwise), both statistical and physiological process models should be adapted, improved and interconnected. Currently, statistical models are often overly disconnected from our biological understanding of tree growth (for example, using linear models for non-linear processes, Fig. \ref{fig:temperaturecomplex}) while process-models are often so complex that they cannot clearly provide testable predictions for empirical data. Together the integration of statistical and more mechanistic process-focused models would provide major insights into the fundamental biology of how tree growth shifts with extended seasons---and yield a unified model for robust predictions of growth responses to warmer, longer seasons across species and levels of warming. % If we want a citation to go with figure ref could use dow2022warm