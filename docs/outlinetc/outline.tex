\documentclass[11pt,letter]{article}
\usepackage[top=1.00in, bottom=1.0in, left=1.1in, right=1.1in]{geometry}
\renewcommand{\baselinestretch}{1.1}
\usepackage{graphicx}
\usepackage{natbib}
\usepackage{amsmath}
\usepackage{todonotes}
\usepackage{hyperref}
\usepackage{parskip}
\usepackage{xcolor}


\def\labelitemi{--}
\parindent=0pt

\begin{document}
\bibliographystyle{/Users/Lizzie/Documents/EndnoteRelated/Bibtex/styles/besjournals}
\renewcommand{\refname}{\CHead{}}

\title{Do growing season length and growth relate? \\ And if not, why not? \\ And if we're not sure, why is that?}
\author{Team Grephon}
\date{\today}
\maketitle

\tableofcontents

\newpage
\section{Outline}

% To do: Get some of the refs to what we have FOUND in our lit review embedded below.
% \textcolor{teal}{This means I have written the text for this part of the outline, I think.}

\begin{enumerate}
\item Introduction (what's happened in the past and where this paper goes)
\begin{enumerate}
\item Multiple fields assume longer GSL means more growth 
\item Unexpected controversy over GSL\footnote{GSL: growing season length;  RL: resource limitation} and tree growth 
\begin{enumerate}
\item But ... Tree ring old studies do show GSL x growth (ring width) ... this might be inferred through tree line or such.
\item Back when: Ecosystem scale growth (NPP?) increases with warming
\item New studies: New studies (all tree rings?) don't show GSL x growth (ring width)
\item Here, we. ...
\begin{enumerate}
\item Hypotheses for why GSL x growth is not found are not equally tested across fields: Constraint issues in provenance but not tree ring etc.
\item Our premise is that some hypotheses for what's going may be tractably already answered by combining data across fields/methods
\item And, you could go far by cross-field tweaking of what each field is doing
\end{enumerate}
\end{enumerate}
\item How warmer temperatures increase tree growth, or not
\begin{enumerate}
\item How they could ...
\begin{enumerate}
\item fundamentally, temperature limits biological processes and is a dominant controller of biological time. 
\item Too cool is bad, too hot is also bad. 
\item These upper limits to rates means absolute time matters also (it's the bottom of the rate equation)
\item temperature should thus limit growth through two major pathways
\begin{enumerate}
\item Directly by increasing rates
\item Through extending periods when development and growth are possible -- extending absolute period of absolute time available (versus relative time)
\item End this section with short part of how well do we know this based on controlled studies? (Alana) ... Maybe end on: So, if the physiological evidence is maybe not so amazing, where does this hypothesis come from?
\end{enumerate}
\end{enumerate}
\end{enumerate}
\item Dendrochronology has long assumed growth decreases with factors that shorten seasons, especially: elevation and latitude
\begin{enumerate}
\item Lots of elevation studies, though most assume the relationship -- fewer show it (see Fig)
\item Many look for shifting climate correlations with elevation, more than growth shifts with elevations ... this is a hallmark of dendro -- tree ring growth as a detector of climate, not other factors that limit growth (usually) 
\item Less work on latitude, but some (see refs)
\item But dendro has never looked deeper, and the literature is really split from the phenology literature -- dendro is conifers mostly and phenology is most often deciduous
\item Note to self: Don't get into complexities here -- the goal is to show the relationship exists at this point in ms. 
\end{enumerate}
\item Given this does seem a real thing, WTF is happening in recent studies? 
\begin{enumerate}
\item Well, they all propose mechanisms (for the most part)
\item Most reported is external drivers --  offsetting increased growth is the latent hypothesis here (so we assume it is happening) -- temperatures are too high or precipitation/drought is limiting
\item In contrast, some also now report brand new fundamental development constraints ... 
\end{enumerate}
\item Yet these hypotheses are tested in radically different ways, never together and miss a whole suite of knowledge on this topic including major possible mechanisms. 
\begin{enumerate}
\item We know this because we systematically reviewed the literature (see Supp for methods) and found it (see FIGURE? Figure instead of text would be great....) .. 
\item The current state of this field is a mess -- while recent papers herald the lack of relationship, we actually found TOTALLY split results suggesting we need a better framework.
\item Part of that means dealing with language, which is not our focus here (but see box)
\item So, what are they? Let us tell you .... % Definitions matter, but only inasmuch as they better define the problem/question (and so are not our focus here)
\end{enumerate}
\end{enumerate}
\end{enumerate}
\item What really could be happening? (Full suite of mechanisms)
\begin{enumerate}
\item Our handy-dandy, fancy-smantzy framework of what controls the relationship between growing season length and plant growth, according to the actual literature that considers dendro, ecology, and plant phys. (I think this line should be one short paragraph). Which has both a suite of external and a suite of internal factors. We highlight what has been studied, and through which disciplinary lenses along the way maybe?
\item External 
\begin{enumerate}
\item External abiotic stuff, which is super well studied by the dendro world. 
\begin{enumerate}
\item Temperature (too high or low; note to self: Save any complexity about this for paths forward)
\item Soil moisture (mainly drought)
\item These above two are measured a lot, because they (1) easy to measure and (2) what dendro likes to look at. 
\item Miscellaneous other (VPD etc.)
\item We found dendro basically always finds external effects ... but that is sort of the AIM of the whole field, no? 
\end{enumerate}
\item Biotic external 
\begin{enumerate}
\item Herbivory
\item Disease
\item Not much on the above too because they are episodic, and less on competition as it is usually considered a nuisance problem for dendro
\item Competition
\end{enumerate}
\end{enumerate}
\item The world of internal programming ... (starting with what is studied, to elephants in the room that are unstudied, as best we can tell)
\begin {enumerate}
\item What has been mentioned ... 
\begin {enumerate}
\item Genetic and developmental constraints include fundamental limits through biophysical (allometry, chemical reaction limits, genetic architecture), but also includes constraints more particular to a certain species, population or individual (e.g., some developmental example -- isn't there one about max photosynthesis depends on temperature early on or such?).  
\item Zohner has suggested his solstice idea ... as universal, but previous work also shows ..
\item Local adaptation can lead to such constraints ... CITES
\item Not mentioned, but likely super important: Evolutionary constraints. These are a legacy of historical evolutionary pressures. Reflects the selection that your ancestral lineages experienced (you're a tree here). Lead to clade differences which are, umm, NEVER discussed in this literature. 
\item Phylogenetic analyses routinely suggest temperature limits show this ... and evergreen versus deciduous findings (Way \& Oren 2010) also suggest it
\end{enumerate}
\item What someone really should have MENTIONED by now 
\begin {enumerate}
\item Plant strategies! Acquisitive to conservative plant strategies (lots of words for this, but basically some species are probably set up to exploit a longer season and some are not ... our indeterminate/determinate discussion goes here) ... This means SPECIES will differ. They also differ in their ... 
\item Growth-reproduction trade-offs, which also lead to between-year dynamics  -- see Hacket-Pain etal 2016
\item The above two, plus phylogeny, mean we should really expect species-differences! Did I mention that? 
\item Actual work on elevation etc. (growth x gsl) is on very limited species list ... we really have never addressed this, perhaps in part because the dominant fields looking at this, dendro and physio, do not think much about species generally or focus on only one species forever and ever. 
\end{enumerate}
\end{enumerate}
\item We have a problem Houston!
\begin {enumerate}
\item The lack of tests of some of the major hypothesis is a problem
\item And the testing of only certain hypotheses in certain disciplines means we lack coherent tests that compare multiple mechanisms. 
\item Too much external in dendro, but endogenous folks don't look at external, so we have no idea of the relative scale of each effect
\item We maybe actually never nailed this in biology.... but there's no time like the present. `Robin! To the bat mobile!'
\end{enumerate}
\item Ay! So much could be happening ... How do we tackle this framework?
\begin{enumerate}
\item Without a cross-discinplinary approach, you cannot tackle this framework. (Lack of standard ways to tackle this question (even when putatively addressing the same question))
\begin{enumerate}
\item People need to do things a little more similarly ...
\item And then we say what are the important explanatory variables, response variables, and give opinions on how they should be measured, etc? Ditto for statistical approaches...
\end{enumerate}
\item Physiologists need to dive deeper on mechanisms so we can compare external and internal drivers better
\begin{enumerate}
\item We have a fundamental lack of mechanistic understanding for when and why growth should increase with GSL and when or why it should not. 
\item Where my physiologists at? We are mostly measuring this growth/phenology stuff without digging deep into mechanisms, we need more interdisciplinary work/cross talk to figure out what is going on when we either do or don't find a link between growing season and growth. Carbohydrate and cell division/expansion dynamics are especially promising.
\item Most ideas are focused on external (dendro), but need to bridge to include and test for constraints
\item this is particularly important if we are going to use these relationships and any constraints we find to project - extrapolating is particularly dangerous when you have the underlying models wrong.
\end{enumerate}
\item This bridge means bridging timescales, from usually short physio to super long dendro
\begin{enumerate}
\item Both teams need to include lag effects of growth, as a nod to the complexity of storage in trees
\item And bring in some new ways to bridge this timescale divide: Measure GSL and tree rings in other ones: FACE, Phynwald, Rainout things, SPRUCE 
\end{enumerate}
\item Dendro needs to embrace internal drivers and blow up its statistical approaches
\begin{enumerate}
\item Tree rings are the answer! These folks have so much to offer, but ...
\item they don't have much phenology data, because they love conifers ...
\item Maybe here: PEP725 x ITRB plot -- sampling overlap; sample more places/species withe phenology data
\item Dendro is designed to see growth through the eyes of climate, looking for other drivers, requires a new outlook, and ...
\item they need to figure out the transfer f(x) that really separates out growth and climate
\item So they need a new sampling design and ... 
\item New stats (see de Sauvage 2022 and probably others where the detrending can really remove stuff)
\item They should more consistently figure out key disturbances throughout the growing season: VPD, baby, and `Yes, maybe the season starts earlier and ends later, but what's happening in the middle? Are there droughts, false springs, intense heat waves?'
\end{enumerate}
\item From these shifts in dendro and physio, it's time to get into comparative numbers folks! (See Box perhaps?)
\begin{enumerate}
\item What actual effect sizes do we expect for GSL versus external drivers? They are actually mushed in the elevation studies ... as is species. 
\item Scale of GSL effect (interannual) versus variation across sites versus species versus drought ... 
\item How do we tell apart high temperatures that accelerate growth from ones that stall it out? Physiologists need to provide some answers here! % Alana has great refs for git issue 1: https://github.com/lizzieinvancouver/grephon/issues/1
\item VPD x temperature curves -- figure those out, include more xylogenesis here
\item They probably also need to thoughtfully include species variation (and include phylogeny in meta-analyses)
\end{enumerate}
\item Both fields need to accept species level variation as variation in growth strategies is a hallmark finding of both life history theory and community ecology
\begin{enumerate}
\item But it's not just `species differ' there are some obvious ways to tackle this
\item Theory suggests aquisitive versus cons trade-off, and reproduction matters -- find variation in this when picking species
\item Take a phylogenetic perspective -- sample thoughtfully across clades when doing the above
\end{enumerate}
\item We need to get a handle on species-level variation because we need to understand the scale of it versus other levels of variation and drivers ... 
\begin{enumerate}
\item Species variation
\item Population within species variation 
\item Interannual variation (where a LOT of work has jumped)
\item What do we know so far? Some studies on this ... deSauvage 2022 Soolanakanay has phenology x growth but not tree ring; Knott et al. 2022 `Phenological response to climate variation in a northern red oak plantation: Links to survival and productivity' has survival and phenological sensitivity but no growth ...  (and King 2013?)
\item And massive opportunity here through common gardens! So core common gardens and estimate this, especially on a yearly basis (not every 5 years) across spp. 
\item Which species to target? Go for ones at the ends of acquitistive conservative ... so Populus, and Oak, for example
\end{enumerate}
\item What we really need to know ... likely means to merge experimental and observational results
\begin{enumerate}
\item Effect of increased growth due to warmth VERSUS due to longer season -- scaling from experiments to forests
\item Experiments testing the effect of GS expansion (at both sides) on growth increments over 2 GS (because autumn warming might increase growth only in the next year if at al). This should be done under favourable conditions (e.g. fully watered conditions) as we already know that drought will stop growth processes.
\item Effect of fruiting (start recording that?)
\item local adaptation vs plasticity in growth strategies 
\item We probably need models to do this, with latent effects to help go from experiment to observational versus using small-scale or other non-realistic experiments to try to set hard bounds on temperature limits etc. 
\end{enumerate}
\end{enumerate}
\item Conclusions
\begin{enumerate}
\item Climate change as rediscovering dusty, old fundamentals, but also possibly things we maybe never figured out
\item Close on why this is important to get right
\begin{enumerate}
\item Carbon storage and climate change
\item Fundamental to physiology, species assembly
\end{enumerate}
\end{enumerate}
\end{enumerate}

\vspace{5ex}
{\bf Box for measurement} Currently not very organized notes, some repetitive
\begin{enumerate}
\item We're not discussing it here in part because it has been discussed elsewhere, and in part because the issue is clearly bigger than measurement ..
\begin{enumerate}
\item While our lit review found a lot of different terms, there was no systematic bias in what was found with what terms (in contrast with suggestion by Koerner) 
\item Ref a table in supp?
\end{enumerate}
\item Very difficult to compare studies as terms are defined differently (e.g., growing season length) ... could compare what the terms mean across fields
\begin{enumerate}
\item Semantic issues make it difficult to know what is or has been tested
\item Everyone is measuring stuff in a slightly different way makes it really hard to compare (especially when everyone is doing only part of the diagram (see figure idea below))
\item Briefly about growing season length: Which to measure? Actual growth end/start as growth (but which growth) or phenological start/end ... 
\item Maybe they are growing and you measured the wrong thing
\item What scale of effect can we detect (and do we expect)?
\item Maybe we are measuring the wrong species ...
\end{enumerate}
\end{enumerate}

{\bf Box for suggested models?}
\begin{enumerate}
\item If you want to resolve this debate, you need comparable estimates -- you have to report similar models 
\item Here's what we suggest?
\end{enumerate}

{\bf Where are we submitting?}

\emph{Nature Climate Change} (3-4K, 4-6 figures), \emph{Global Change Biology}, \emph{New Phytologist} ... thinking of places where we will get an interdisciplinary audience.\\

{\bf Methods}: Found 33 refs through ISI search on 12 April 2023: "growing season length" AND "tree ring*" (ALL FIELDS). We then looked through citations within and cited since, as well as incidental during the review process.\\


\section{Figure ideas}

\begin{enumerate}
\item Ruben's figure
\item Alana's rate x temperature (x limitations) figure: maybe add in agriculture
\item Conceptual of connections (sort of path diagram figure)
\begin{enumerate}
\item Start simple: temp $\rightarrow$ GSL $\rightarrow$ growth (this was NEP x growth studies from a while ago; and the tree ring people go backwards along this diagram)
\item Next: Just the conceptual: temp $\rightarrow$ GSL $\rightarrow$ growth PLUS resource bubble
\item Supp figure on measurements? Figure with methods layered on, maybe do just for start and end of GSL or such
\item Layer onto this figure: What each fields offer -- size of arrow is the number of studies that do it
\item Layer onto this figure: What each field could offer if they did x, y, z -- FUTURE directions also
\item We need relative magnitude of these arrows
\item Message here: Full path diagram of what is happening is more complicated than perhaps is being let on, and no one has tested it fully
\begin{enumerate}
\item The provenance people have a lot to offer
\ but they don't measure annual growth ... they have the constraint information, but asking it for a different reason. 
\item Could they manipulate temp and resource?
\item Or we need the tree ring people to work on this: tree rings across distribution  $\rightarrow$ response + variability of response
\item Greenhouse/growth chamber folks can do the full figure
\end{enumerate}
% \item Figure of ideal workflow?
\end{enumerate}
\item ISI cross-pollination currently across fields or see \url{https://www.connectedpapers.com/}
\begin{enumerate}
\item tree rings in climatology
\item tree rings in ecology
\item constraint folks
\item forestry plots (provenance trials)
\item experiments ... 
\end{enumerate}
\item PEP725 x ITRB plot
\item Figure for future part?
\item Table/figure on advantages/limitations for each approach? Key places where interdisciplinary opportunities (leverage)
\end{enumerate}

\newpage
{\bf Stuff in need of a home in outline, maybe} 
\begin{enumerate}
\item This is important! 
\begin{enumerate}
\item Carbon storage and climate change
\item Fundamental to physiology, species assembly
\end{enumerate}
\item Basics: GDD model of growth suggests GSL x growth (with some base temperature for GDD)
\item Test for possible endogenous drivers with flux towers? (To bridge some of the current method x endo/evo divide)
\item atitude or altitude vs growth. Would be cool to actually demonstrate this
\item The amount of growth that is happening at each point in the growing season and how is it related to GSL. The question I have here is if earlier GS means that peak growth will happen early, then if maybe poor conditions later in the GS don’t matter as much, or maybe they do? Or if there are endogenous constraints on the peaking of growth then shifts to earlier GS wouldn’t really help at all
\item We’ve talked about this before but collecting cores from places where phenology has been already recorded and is continuing to be recorded would be really important for examining long term changes and trends. It seems to me like collecting cores would be more feasible than collecting phenology data (which you’d need to visit every location multiple times)
\item Also still really curious about where the carbon goes. How much of it goes to growth vs maintenance or reproduction or whatever else. Would be cool to track it in flux tower areas where they can detect the CO2 drawdown across the season showing that photosynthesis is happening earlier but if we’re not detecting radial growth then what is the consequence of this earlier photosynthesis start. Another way to look at this could be an experiment where we would tag CO2 (can you tag CO2? maybe isotopes) and then destructively sample the plants at different points along the growing season to find out where the CO2 went? Of course the caveat being that it would likely be juveniles, or model organisms so how much would we really be able to extrapolate?
\end{enumerate}




{\bf Things that need a home}
\begin{enumerate}
\item Conifers (tree ring data) versus deciduous (phenology data)
\item Conifers: Does leafout matter in conifers or would it be much more related to when they start photosynthesizing with old leaves?
\item Species diversity in tree ring studies ... maybe make table on whether the studies with tree rings and growth have looked at dominant canopy species \todo{add to Janneke's table?}
\item Maybe ... which species have budset constraints been shown in?
\end{enumerate}


\newpage
\section{The table is done in late June!}

\emph{Remember what our main new aims for the paper:}
We're interested in constraints, resource limitations, species and interdisciplinarity; we should stay focused on this. Species too... 

\emph{What we have found overall (3 July 2023 meeting):}
\begin{enumerate}
\item Lizzie reported out on the question of which studies do or don't find relationships. Seems like of the common growth metrics, annual cores do not find our definition and they don't always find their definition either.  No relationships with country or biome jump out (pretty biased towards certain places though). We need to do some more work to look at species as that looks complicated,  but nothing jumped out in which species do or do not show relationships. 
\item Team external: lots of tree ring studies look at external and they all find a relationship. Not many experiments do and whether they find a relationship or not is more mixed than for tree rings. Next steps for this could be that the experimental and forest-scale results need to connect better: maybe this means looking at better metrics than temp and precip (such as PET) or looking at interactions or TBD.
\item Team endogenous found three major types of effects: provenance, species (and functional types) and 'growing early' (Zohner paper bolus; though Frederik only mentioned Zani paper: `Greater GPP, higher Photosynthesis, earlier growing season leads to earlier senescence (Zani 2020)'). Very few studies mechanistically try to understand what causes provenance, species effects. (Of the 56 rows in our table 23 indicate that authors have looked and ~ 21 found evidence for endogenous factors. These are mostly provenance / experimental studies, hardly any tree ring studies.  )
\begin{enumerate}
\item Maybe we (or physiologists) need to review what the mechanisms could be? Ideas are... 
\begin{enumerate}
\item Leaf life span
\item Tade-offs between reproduction and growth
\item budset programming
\item Proportion of determinate vs. indeterminate buds. Can we get info on that?
\item Maybe group species by CSR or shade tolerance?
\item Check out Silvics manual (\url{https://www.srs.fs.usda.gov/pubs/misc/ag_654/table_of_contents.htm})
\item It could be nice to discuss the tradeoffs involved of a tree’s growing season ending. Potential gains in growth, fruit production and reserve allocation vs. Tissue damages and fitness loss (unripened fruits, dieback), loss of nutrients in foliage due to delayed senescence.
\end{enumerate}
\item Whatever it is, we need to somehow see it scale up to latitudinal variation (provenance effects); this reality seems to be missing from Zohner and other work
\end{enumerate}
\item Next steps from Frederik: It seems there is not much to get out of the table regarding endogenous factors. If we would like to include a section in the paper about this I suggest we gather and discuss the current hypothesis around this topic. What comes into my mind is: amortization time, reach a certain reserve level, distinguish between determinate and indeterminate growth, photoperiodic constraints and induction of bud set, senescence and dormancy. I would be interested to do that!
\end{enumerate}


Also, Lizzie checked out Rossi \emph{et al.} 2013 (cited in Korner paper from 2023) and it does not appear relevant to our table. See \url{https://link.springer.com/article/10.1007/s00442-006-0625-7} if you want to check for yourself. 


\newpage
\section{Where to find other notes}
\verb|grephon/notes/2023SpringLucidEtc/Grephon_Lucid board notes - Google Docs.pdf|  is a file extracting the main take-homes from our Lucidboard work (December 2022 until sometime in early 2023). You can also see the Lucid board in PDF file: Grephon December 2022 brainstorm

See treeringnotes.pdf for our notes from reading papers in fall 2022. 


In August 2023, Lizzie reviewed this doc and the Lucid board (again) and a little the notes from last year, she pulled out the following:

Stuff to probably work into ms ...
\begin{itemize}
\item Biophysical constraints
\begin{itemize}
\item To what extent can the internal phenological program constrain C uptake/additional growth and influence the fate of C?
\item Do phenological shifts push the growing season into periods of low VPD? This should enhance growth if turgor is the limit, but might lead to more regional variation in response. Can we test this with treeNet data?
\end{itemize}
\item How much local adaptation is there in G x P?
\item Show PEP725 + ITRB overlay (w/ Rubén) and highlight potential of this data in understanding growth-phenology
\item Stuff we critically need to know
\begin{itemize}
\item When does growth happen for different species (early versus late season) and how flexible is this pattern?
\item How photoperiod matters (or doesn't and why folks get it wrong in dendro so much)
\item Better understanding of NSCs and phenology - are we seeing differences across wood anatomy or is it even more species specific or population specific
\item Differences in patterns across space/species/source populations?
\end{itemize}
\item It may not be a long-term stable strategy to try to adjust growth dramatically year-to-year, so should we really expect this correlation?
\item Cite Knott et al. 2022 `Phenological response to climate variation in a northernred oak plantation: Links to survival and productivity' -- no growth data, we really need more common garden studies
\end{itemize}

Especially cool ideas to do someday (but maybe not in this paper)
\begin{itemize}
\item What does the flux tower data tell us about the availability of photosynthate for growth? Can we use this to make assumptions about the sugars going elsewhere, like to heartwood?
\item Compare numbers: ring growth across elevations/across time/in dendro studies/dendrometers
\item What is the regional variation like in the growth phenology relationship? Are there places where growth is clearly increasing? If so, what is the climate like? Where would we expect growth to increase with a longer GS? Where would we not?
\item Statistical issues
\end{itemize}


\end{document}



{\bf Wait, WTF is a constraint? Or is there a gradient of constraint to RL ... }
\begin{enumerate}

\item Lifespan determination growth mush
\begin{enumerate}
\item Seasonal growth stops: Budset, Zohner equinox paper
\item Does high temperature fit here? (Definitely on a different timescale)
\end{enumerate}
\item High temperature limitation -- enzymatically it's over
\item Low temperature limitation -- sort of like energy limitation 
\item Things we feel sure are resource limitation
\begin{enumerate}
\item Nutrients
\item Water
\end{enumerate}
\item Terms we could use. ... 
\begin{enumerate}
\item Phenomenological without a real mechanism (plants stop growing at the same time every year)
\item External versus internal
\item Abiotic versus biotic
\end{enumerate}
\end{enumerate}
