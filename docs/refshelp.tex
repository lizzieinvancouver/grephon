I need help with refs:
Dendrochronology---the study of tree rings and tree ring dating---has long assumed growth decreases with shorter seasons (CITES), though this connection is almost made across space, not time (CITES).
 Elevation and latitude---two major factors that generally shorten seasons, and change a suite of climatic factors---generally lead to less annual growth (CITES)
 Most current work in dendrochronology assumes this relationship and focuses more on how climatic factors shift across these gradients (CITES)
 
 This focus on climatic signals in tree growth (measured by ring width) is the hallmark of dendrochronology, with the methods in the field generally designed around this aim. (RUBENCITES)

many tree ring studies have shown both temperature extremes correlate with lower growth (CITES). These temperature limits, however, generally assume sufficient soil moisture for plant growth---and a suite of tree ring studies confirm this finding correlations with precipitation (CITES) or other metrics related to plant access to water (CITES). The actual relationship between temperature, moisture and tree growth is more complex, as studies finding strong correlations between vapor pressure deficit and growth attest (CITES). 

KavyaAlana: ref for NDVI and herbivory (discussed last fall: Hypothesis from Alana: EOS10 is just a measure of herbivory ... so maybe this paper just means that earlier springs means earlier onset of 10 percent herbivory)? 

Example of how max photosynthetic rate depends on early development temperature or something? 

Add to TO DO: Work on methods for lit review in SUPP
- Be sure to say studies perform dozens to sometimes of tests so we did not include a row for each one, instead we aimed for unique studies or tests within a paper. 