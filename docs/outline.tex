\documentclass[11pt,letter]{article}
\usepackage[top=1.00in, bottom=1.0in, left=1.1in, right=1.1in]{geometry}
\renewcommand{\baselinestretch}{1.1}
\usepackage{graphicx}
\usepackage{natbib}
\usepackage{amsmath}
\usepackage{todonotes}


\def\labelitemi{--}
\parindent=0pt

\begin{document}
\bibliographystyle{/Users/Lizzie/Documents/EndnoteRelated/Bibtex/styles/besjournals}
\renewcommand{\refname}{\CHead{}}

\title{Let's answer when/how/why growing season length and growth relate!}
\author{Team Grephon}
\date{\today}
\maketitle

\tableofcontents


\section{When the table is done in May}

\emph{Remember what our main new aims for the paper:}
We're interested in constraints, resource limitations, species and interdisciplinarity; we should stay focused on this.

\begin{itemize}
\item Interdisciplinary needed, especially to address hypotheses of resource limitation versus constraint
\begin{itemize}
\item Semantic issues make it difficult to know what is or has been tested
\item Everyone is measuring stuff in a slightly different way makes it really hard to compare (especially when everyone is doing only part of the diagram)
\item Boring thing about growing season length: Which to measure? Actual growth end/start as growth (but which growth) or phenological start/end ... 
\end{itemize}
\item Full path diagram of what is happening is more complicated than perhaps is being let on, and no one has tested it fully
\begin{itemize}
\item Path diagram figure
\begin{itemize}
\item Start simple: temp $\rightarrow$ GSL $\rightarrow$ growth (this was NEP x growth studies from a while ago; and the tree ring people go backwards along this diagram)
\item Next: Just the conceptual: temp $\rightarrow$ GSL $\rightarrow$ growth PLUS resource bubble
\item Supp figure on measurements? Figure with methods layered on, maybe do just for start and end of GSL or such
\item Layer onto this figure: What each fields offer -- size of arrow is the number of studies that do it
\item Layer onto this figure: What each field could offer if they did x, y, z -- FUTURE directions also
\item We need relative magnitude of these arrows
\end{itemize}
\item Figure of ideal workflow?
\item Figure of concepts versus methods
\item Okay, how would you test the full workflow
\end{itemize}
\item The provenance people have a lot to offer
\ but they don't measure annual growth ... they have the constraint information, but asking it for a different reason. 
\item Could they manipulate temp and resource?
\item Or we need the tree ring people to work on this: tree rings across distribution  $\rightarrow$ response + variability of response
\item Greenhouse/growth chamber folks can do the full figure
\item Species! Species differences -- keep thinking about where to fit this in; can we look at representativity across tree ring / phenology / experiments (mention competition in future directions; we have to get a handle on this conceptual diagram first)
\end{itemize}

\section{What to do next or some day ...}

\begin{enumerate}
\item Review our old notes and outlines
\item Fill out table Janneke is making ... what do we want out of it?
\item How to review the papers (pick them) for Janneke's table
\item Measurement comparisons: growth estimated from tree rings (climatology), tree rings (ecology), tree rings in experiments or such?
\item Why is the Parent et al. 2012 curve different than Alan's curves for $A_{net}$?
\end{enumerate}


\section{Outline}

Translator:
\begin{enumerate}
\item GSL: growing season length
\item RL: resource limitation
\end{enumerate}

\begin{enumerate}
\item Introduction (what's happened in the past and where this paper goes)
\begin{enumerate}
\item Unexpected controversy over GSL and tree growth \todo{needs organizing}
\begin{enumerate}
\item Basics: GDD model of growth suggests GSL x growth (with some base temperature for GDD)
\item Back when: Ecosystem scale growth (NPP?) increases with warming
\item New studies: Tree rings don't show GSL x growth (ring width)
\item But ... Tree ring old studies do show GSL x growth (ring width) \todo{Do they? Need refs}
\end{enumerate}
\item What's going on?! (Briefly)
\begin{enumerate}
\item Measurement issues 
\item Biophysical constraints
\item Resource limitations
\end{enumerate}
\item Here, we. ...
\begin{enumerate}
\item Our premise is that some hypotheses for what's going may be tractably already answered by combining data across fields/methods
\item Is our premise also that constraint issues are not well included in current hypotheses?
\item And, you could go far by cross-field tweaking of what each field is doing
\end{enumerate}
\item This is important! \todo{fit this into intro}
\begin{enumerate}
\item Carbon storage and climate change
\item Fundamental to physiology, species assembly
\end{enumerate}
\end{enumerate}
\item Section: Review three reasons for not growing (Our opinion on now)
\begin{enumerate} 
\item Overview paragraph of three reasons
\begin{enumerate} 
\item measurement -- see box  (include measurement only here, maybe)
\item Resource limitation
\item Constraints
\end{enumerate}
\item Resource limitation, evidence for an against \todo{we need to go back to literature to work on this}
\begin{enumerate}
\item Nutrients
\item Water
\item Is this more species-specific?
\end{enumerate}
\item Constraints, evidence for an against \todo{we need to go back to literature to work on this}
\begin{enumerate}
\item Leaf life span
\item Budset stuff ... (Zohner, Sool.)
\item Evidence across species? Or which is species-specific
\end{enumerate}
\end{enumerate}
\item What do do next (The future! Is there a framework to our future directions? It would be nice if we found one)
\begin{enumerate}
\item Tree ring people should ... get better phenology data and a better sampling design ... and figure out the transfer f(x)
\begin{enumerate}
\item core trials with phenology data
\item PEP725 x ITRB plot -- sampling overlap; sample more places/species withe phenology data
\item Take more fine-scale measurements
\end{enumerate}
\item Permanent plot data (provenance/forestry plots) need some temporal resolution and need more allometry data
\begin{enumerate}
\item Do they have GSL? 
\item They calculate growth every 5 years, they need annual data!
\item Allometry here is big opportunity: measuring at different heights (more fine-scale data)
\end{enumerate}
\item Greenhouse/chamber experiments
\begin{enumerate}
\item Our experiment (which needs a name)!
\item VPD x temperature curves -- figure those out, include more xylogenesis here
\end{enumerate}
\item Big experiments
\begin{enumerate}
\item You could do one for this topic
\item Measure GSL and tree rings in other ones: FACE, Phynwald, Rainout things, SPRUCE 
\end{enumerate}
\item Constraint experiments (yes, messy topic ... not sure where all this goes ...)
\begin{enumerate}
\item Give plants everything they want and show they still shut down growth at the same time (this goes with some full factorial experiments with drought etc. so we could estimate when the constrain matters ... we could say 'of course there is a constraint but we need to better understand when it would matter with climate change)
\item How do people show contraints?
\begin{enumerate}
\item You sort of need the mechanism ... molecular, experiment on proximate cues etc.
\item Show variation across latitude often for this
\end{enumerate}
\end{enumerate}
\end{enumerate}
\end{enumerate}

{\bf Feelings \& thoughts coming up}
\begin{enumerate}
\item If you want to resolve this debate, you need comparable estimates -- you have to report similar models (you % Janneke being positive 
\item Some of the fields (tree rings, where we found so far studies do not look consistently at growing season length x growth, they are more interested in spring temperatures x growth) do not care as much about the mechanism... but it really matters
\item Very difficult to compare studies as terms are defined differently (e.g., growing season length) ... could compare what the terms mean across fields
\item Two major ways to measure growing season length: phenological and growth data (xylogenesis)
\end{enumerate}

{\bf Wait, WTF is a constraint? Or is there a gradient of constraint to RL ... }
\begin{enumerate}

\item Lifespan determination growth mush
\begin{enumerate}
\item Seasonal growth stops: Budset, Zohner equinox paper
\item Does high temperature fit here? (Definitely on a different timescale)
\end{enumerate}
\item High temperature limitation -- enzymatically it's over
\item Low temperature limitation -- sort of like energy limitation 
\item Things we feel sure are resource limitation
\begin{enumerate}
\item Nutrients
\item Water
\end{enumerate}
\item Terms we could use. ... 
\begin{enumerate}
\item Phenomenological without a real mechanism (plants stop growing at the same time every year)
\item External versus internal
\item Abiotic versus biotic
\end{enumerate}
\item Kavya adds: I was thinking a little more about the whole constraints vs. limitation thing and this is a thought I had:
Is the distinction between resource limit and constraint that resource limitation occurs across a spectrum while constraints have more rigid boundaries (even for the same organism)? In this case (and following Alana’s cool figure), temp could be both — at the two ends of the spectrum it's a constraint (cannot photosynthesize below or above a threshold (enzymatic limits)) but in the middle, it’s a resource limitation? Or maybe up till Topt it a resource limit and beyond that it becomes a constraint?
Also I think it would be good for us to distinguish between net C uptake and allocation to growth? As in trees might still be photosynthesizing but not growing, or not growing in the ways we think they would. Or maybe this just goes into the measurement box with the idea that we aren't measuring the correct/all of the facets of growth. 
\item Janneke then added: problem is that temperature influences the rate of everything to do with growth and resource uptake (e.g., water and nitrogen) and is also damaging agent (e.g., frost, denaturing)
\end{enumerate}


{\bf Things that need a home}
\begin{enumerate}
\item Patterns of GSL x growth across elevation/latitude
\item What is growing season length? (Actually when growth starts? Or something else?) 
\item Measurement issues
\begin{enumerate}
\item Maybe they are growing and you measured the wrong thing
\item What scale of effect can we detect (and do we expect)?
\item Maybe we are measuring the wrong species ...
\end{enumerate}
\item Conifers (tree ring data) versus deciduous (phenology data)
\item Conifers: Does leafout matter in conifers or would it be much more related to when they start photosynthesizing with old leaves?
\item Species diversity in tree ring studies ... maybe make table on whether the studies with tree rings and growth have looked at dominant canopy species \todo{add to Janneke's table?}
\item Maybe ... which species have budset constraints been shown in?
\end{enumerate}

\section{Figure ideas}

\begin{enumerate}
\item Ruben's figure
\item Alana's rate x temperature (x limitations) figure: maybe add in agriculture
\item ISI cross-pollination currently across fields
\begin{enumerate}
\item tree rings in climatology
\item tree rings in ecology
\item constraint folks
\item forestry plots (provenance trials)
\item experiments ... 
\end{enumerate}
\item PEP725 x ITRB plot
\item Figure for future part?
\item Table/figure on advantages/limitations for each approach? Key places where interdisciplinary opportunities (leverage)
\end{enumerate}


\end{document}

\begin{enumerate}
\item
\end{enumerate}