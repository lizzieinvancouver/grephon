\documentclass[11pt,letter]{article}
\usepackage[top=1.00in, bottom=1.0in, left=1.1in, right=1.1in]{geometry}
\renewcommand{\baselinestretch}{1.1}
\usepackage{graphicx}
\usepackage{natbib}
\usepackage{amsmath}
\usepackage{todonotes}
\usepackage{hyperref}
\usepackage{parskip}


\def\labelitemi{--}
\parindent=0pt

\begin{document}
\bibliographystyle{/Users/Lizzie/Documents/EndnoteRelated/Bibtex/styles/besjournals}
\renewcommand{\refname}{\CHead{}}

\title{Do growing season length and growth relate? \\ And if not, why not? \\ And if we're not sure, why is that?}
\author{Team Grephon}
\date{\today}
\maketitle

\tableofcontents

\section{The table is done in late June!}

\emph{Remember what our main new aims for the paper:}
We're interested in constraints, resource limitations, species and interdisciplinarity; we should stay focused on this. Species too... 

\emph{What we have found overall (3 July 2023 meeting):}
\begin{enumerate}
\item Lizzie reported out on the question of which studies do or don't find relationships. Seems like of the common growth metrics, annual cores do not find our definition and they don't always find their definition either.  No relationships with country or biome jump out (pretty biased towards certain places though). We need to do some more work to look at species as that looks complicated,  but nothing jumped out in which species do or do not show relationships. 
\item Team external: lots of tree ring studies look at external and they all find a relationship. Not many experiments do and whether they find a relationship or not is more mixed than for tree rings. Next steps for this could be that the experimental and forest-scale results need to connect better: maybe this means looking at better metrics than temp and precip (such as PET) or looking at interactions or TBD.
\item Team endogenous found three major types of effects: provenance, species (and functional types) and 'growing early' (Zohner paper bolus; though Frederik only mentioned Zani paper: `Greater GPP, higher Photosynthesis, earlier growing season leads to earlier senescence (Zani 2020)'). Very few studies mechanistically try to understand what causes provenance, species effects. (Of the 56 rows in our table 23 indicate that authors have looked and ~ 21 found evidence for endogenous factors. These are mostly provenance / experimental studies, hardly any tree ring studies.  )
\begin{enumerate}
\item Maybe we (or physiologists) need to review what the mechanisms could be? Ideas are... 
\begin{enumerate}
\item Leaf life span
\item Tade-offs between reproduction and growth
\item budset programming
\item Proportion of determinate vs. indeterminate buds. Can we get info on that?
\item Maybe group species by CSR or shade tolerance?
\item Check out Silvics manual (\url{https://www.srs.fs.usda.gov/pubs/misc/ag_654/table_of_contents.htm})
\item It could be nice to discuss the tradeoffs involved of a tree’s growing season ending. Potential gains in growth, fruit production and reserve allocation vs. Tissue damages and fitness loss (unripened fruits, dieback), loss of nutrients in foliage due to delayed senescence.
\end{enumerate}
\item Whatever it is, we need to somehow see it scale up to latitudinal variation (provenance effects); this reality seems to be missing from Zohner and other work
\end{enumerate}
\item Next steps from Frederik: It seems there is not much to get out of the table regarding endogenous factors. If we would like to include a section in the paper about this I suggest we gather and discuss the current hypothesis around this topic. What comes into my mind is: amortization time, reach a certain reserve level, distinguish between determinate and indeterminate growth, photoperiodic constraints and induction of bud set, senescence and dormancy. I would be interested to do that!
\end{enumerate}


Also, Lizzie checked out Rossi \emph{et al.} 2013 (cited in Korner paper from 2023) and it does not appear relevant to our table. See \url{https://link.springer.com/article/10.1007/s00442-006-0625-7} if you want to check for yourself. 






\section{Outline}

\begin{enumerate}
\item Introduction (what's happened in the past and where this paper goes)
\begin{enumerate}
\item Multiple fields assume longer GSL means more growth 
\item Unexpected controversy over GSL\footnote{GSL: growing season length;  RL: resource limitation} and tree growth 
\begin{enumerate}
\item But ... Tree ring old studies do show GSL x growth (ring width) ... this might be inferred through tree line or such.
\item Back when: Ecosystem scale growth (NPP?) increases with warming
\item New studies: New studies (all tree rings?) don't show GSL x growth (ring width)
\item Here, we. ...
\begin{enumerate}
\item Hypotheses for why GSL x growth is not found are not equally tested across fields: Constraint issues in provenance but not tree ring etc.
\item Our premise is that some hypotheses for what's going may be tractably already answered by combining data across fields/methods
\item And, you could go far by cross-field tweaking of what each field is doing
\end{enumerate}
\end{enumerate}
\item How warmer temperatures increase tree growth, or not
\begin{enumerate}
\item How they could ...
\begin{enumerate}
\item fundamentally, temperature limits biological processes and is a dominant controller of biological time. 
\item Too cool is bad, too hot is also bad. 
\item These upper limits to rates means absolute time matters also (it's the bottom of the rate equation)
\item temperature should thus limit growth through two major pathways
\begin{enumerate}
\item Directly by increasing rates
\item Through extending periods when development and growth are possible -- extending absolute period of absolute time available (versus relative time)
\end{enumerate}
\end{enumerate}
\item Where do we see this? Trends over elevation and trends over latitude (??)
\end{enumerate}
\item Why this might not happen...
\begin{enumerate}
\item Measurement issues 
\item Biophysical constraints (endogenous)
\item Resource limitations (exogenous) 
\end{enumerate}
\end{enumerate}
\item Section: Review three reasons for not growing 
\begin{enumerate} 
\item Overview paragraph of three reasons
\begin{enumerate} 
\item Measurement -- see box/figure  (include measurement only here or briefly so we move through it fast; Table results: What metrics of growth? could fit here)
\item Resource limitation (exogenous)
\item Constraints
\end{enumerate}
\item Set up table results: 
\begin{enumerate}
\item Which fields are tackling this question?
\item People are not measuring the same thing
\end{enumerate}
\item Resource limitation, evidence for an against \todo{Table can help us with this?}
\begin{enumerate}
\item Nutrients
\item Water
\item Is this more species-specific?
\end{enumerate}
\item Constraints, evidence for an against \todo{Table can help us with this?}
\begin{enumerate}
\item Leaf life span
\item Budset stuff ... (Zohner, Sool.)
\item Evidence across species? Or which is species-specific
\end{enumerate}
\item Endo vs exo focus breaks down by field
\end{enumerate}
\item Tying it altogether/where do we go next?
\begin{enumerate}
\item Climate change as rediscovering dusty, old fundamentals, but also possibly things we maybe never figured out
\item Growth x reproduction trade-off
\item No mention of fruiting/reproduction in the papers we reviewed, right?
\item local adaptation vs plasticity in growth strategies 
\item what are the effect sizes of growth x GSL in the old literature (across elevational gradients; did they also do across latitude?) versus effect sizes within a site 
\item Definitions matter, but only inasmuch as they better define the problem/question (and so are not our focus here)
\item Lag effects of growth and the complexity of storage in trees
\end{enumerate}
\item Future work should ...
\begin{enumerate}
\item Look into local adaptation vs plasticity in growth strategies
\item What actual effect sizes do we expect within site across years for different species?
\item Stop schmering across species. Think and measure stuff about species that matters?
\begin{enumerate}
\item Determinate vs indeterminate buds
\item Successional stage
\item Phylogeny? 
\item Functional groups? (This would merge to ecosystem models.)
\end{enumerate}
\item Endogenous factors we need to study: 
\begin{enumerate}
\item Local adaptation in GSL x growth patterns (budset? budburst? this is not yet well formulated)
\item species are more or less limited by plastic vs. local adaptation to growth
\item So core common gardens and estimate this, especially on a yearly basis (not every 5 years) across spp. 
\item Leaf longevity?
\end{enumerate}
\item Test for possible endogenous drivers with flux towers? (To bridge some of the current method x endo/evo divide)
\end{enumerate}
\end{enumerate}


{\bf Stuff in need of a home in outline} 
\begin{enumerate}
\item This is important! 
\begin{enumerate}
\item Carbon storage and climate change
\item Fundamental to physiology, species assembly
\end{enumerate}
\item Basics: GDD model of growth suggests GSL x growth (with some base temperature for GDD)
\end{enumerate}




{\bf Where are we submitting?}

\emph{Nature Climate Change} (3-4K, 4-6 figures), \emph{Global Change Biology}, \emph{New Phytologist} ... thinking of places where we will get an interdisciplinary audience.\\

{\bf Methods}: Found 33 refs through ISI search on 12 April 2023: "growing season length" AND "tree ring*" (ALL FIELDS). We then looked through citations within and cited since, as well as incidental during the review process.\\

\newpage
{\bf Stuff that might work for the future directions}\\
\emph{Plan is to do the table, then work with fresh heads on the future directions .. THEN come back and look at these notes}
\begin{enumerate}
\item If you want to resolve this debate, you need comparable estimates -- you have to report similar models % Janneke being positive 
\item Some of the fields (tree rings, where we found so far studies do not look consistently at growing season length x growth, they are more interested in spring temperatures x growth) do not care as much about the mechanism... but it really matters
\item Very difficult to compare studies as terms are defined differently (e.g., growing season length) ... could compare what the terms mean across fields
\begin{enumerate}
\item Semantic issues make it difficult to know what is or has been tested
\item Everyone is measuring stuff in a slightly different way makes it really hard to compare (especially when everyone is doing only part of the diagram (see figure idea below))
\item Briefly about growing season length: Which to measure? Actual growth end/start as growth (but which growth) or phenological start/end ... 
\end{enumerate}
\item Two major ways to measure growing season length: phenological and growth data (xylogenesis)
\end{enumerate}

\begin{enumerate}
\item Tree ring people should ... get better phenology data and a better sampling design ... and figure out the transfer f(x)
\begin{enumerate}
\item core trials with phenology data
\item PEP725 x ITRB plot -- sampling overlap; sample more places/species withe phenology data
\item Take more fine-scale measurements
\end{enumerate}
\item Permanent plot data (provenance/forestry plots) need some temporal resolution and need more allometry data
\begin{enumerate}
\item Do they have GSL? 
\item They calculate growth every 5 years, they need annual data!
\item Allometry here is big opportunity: measuring at different heights (more fine-scale data)
\end{enumerate}
\item Greenhouse/chamber experiments
\begin{enumerate}
\item Our experiment (which needs a name)!
\item VPD x temperature curves -- figure those out, include more xylogenesis here
\end{enumerate}
\item Big experiments
\begin{enumerate}
\item You could do one for this topic
\item Measure GSL and tree rings in other ones: FACE, Phynwald, Rainout things, SPRUCE 
\end{enumerate}
\item Constraint experiments (yes, messy topic ... not sure where all this goes ...)
\begin{enumerate}
\item Give plants everything they want and show they still shut down growth at the same time (this goes with some full factorial experiments with drought etc. so we could estimate when the constrain matters ... we could say 'of course there is a constraint but we need to better understand when it would matter with climate change)
\item How do people show constraints?
\begin{enumerate}
\item You sort of need the mechanism ... molecular, experiment on proximate cues etc.
\item Show variation across latitude often for this
\end{enumerate}
\end{enumerate}
\item Species! Species differences -- keep thinking about where to fit this in; can we look at representativity across tree ring / phenology / experiments (mention competition in future directions; we have to get a handle on this conceptual diagram first)
\end{enumerate}


{\bf Wait, WTF is a constraint? Or is there a gradient of constraint to RL ... }
\begin{enumerate}

\item Lifespan determination growth mush
\begin{enumerate}
\item Seasonal growth stops: Budset, Zohner equinox paper
\item Does high temperature fit here? (Definitely on a different timescale)
\end{enumerate}
\item High temperature limitation -- enzymatically it's over
\item Low temperature limitation -- sort of like energy limitation 
\item Things we feel sure are resource limitation
\begin{enumerate}
\item Nutrients
\item Water
\end{enumerate}
\item Terms we could use. ... 
\begin{enumerate}
\item Phenomenological without a real mechanism (plants stop growing at the same time every year)
\item External versus internal
\item Abiotic versus biotic
\end{enumerate}
\item Kavya adds: I was thinking a little more about the whole constraints vs. limitation thing and this is a thought I had:
Is the distinction between resource limit and constraint that resource limitation occurs across a spectrum while constraints have more rigid boundaries (even for the same organism)? In this case (and following Alana’s cool figure), temp could be both — at the two ends of the spectrum it's a constraint (cannot photosynthesize below or above a threshold (enzymatic limits)) but in the middle, it’s a resource limitation? Or maybe up till Topt it a resource limit and beyond that it becomes a constraint?
Also I think it would be good for us to distinguish between net C uptake and allocation to growth? As in trees might still be photosynthesizing but not growing, or not growing in the ways we think they would. Or maybe this just goes into the measurement box with the idea that we aren't measuring the correct/all of the facets of growth. 
\item Janneke then added: problem is that temperature influences the rate of everything to do with growth and resource uptake (e.g., water and nitrogen) and is also damaging agent (e.g., frost, denaturing)
\end{enumerate}


{\bf Things that need a home}
\begin{enumerate}
\item Patterns of GSL x growth across elevation/latitude
\item What is growing season length? (Actually when growth starts? Or something else?) 
\item Measurement issues
\begin{enumerate}
\item Maybe they are growing and you measured the wrong thing
\item What scale of effect can we detect (and do we expect)?
\item Maybe we are measuring the wrong species ...
\end{enumerate}
\item Conifers (tree ring data) versus deciduous (phenology data)
\item Conifers: Does leafout matter in conifers or would it be much more related to when they start photosynthesizing with old leaves?
\item Species diversity in tree ring studies ... maybe make table on whether the studies with tree rings and growth have looked at dominant canopy species \todo{add to Janneke's table?}
\item Maybe ... which species have budset constraints been shown in?
\end{enumerate}

\section{Figure ideas}

\begin{enumerate}
\item Ruben's figure
\item Alana's rate x temperature (x limitations) figure: maybe add in agriculture
\item Conceptual of connections (sort of path diagram figure)
\begin{enumerate}
\item Start simple: temp $\rightarrow$ GSL $\rightarrow$ growth (this was NEP x growth studies from a while ago; and the tree ring people go backwards along this diagram)
\item Next: Just the conceptual: temp $\rightarrow$ GSL $\rightarrow$ growth PLUS resource bubble
\item Supp figure on measurements? Figure with methods layered on, maybe do just for start and end of GSL or such
\item Layer onto this figure: What each fields offer -- size of arrow is the number of studies that do it
\item Layer onto this figure: What each field could offer if they did x, y, z -- FUTURE directions also
\item We need relative magnitude of these arrows
\item Message here: Full path diagram of what is happening is more complicated than perhaps is being let on, and no one has tested it fully
\begin{enumerate}
\item The provenance people have a lot to offer
\ but they don't measure annual growth ... they have the constraint information, but asking it for a different reason. 
\item Could they manipulate temp and resource?
\item Or we need the tree ring people to work on this: tree rings across distribution  $\rightarrow$ response + variability of response
\item Greenhouse/growth chamber folks can do the full figure
\end{enumerate}
% \item Figure of ideal workflow?
\end{enumerate}
\item ISI cross-pollination currently across fields or see \url{https://www.connectedpapers.com/}
\begin{enumerate}
\item tree rings in climatology
\item tree rings in ecology
\item constraint folks
\item forestry plots (provenance trials)
\item experiments ... 
\end{enumerate}
\item PEP725 x ITRB plot
\item Figure for future part?
\item Table/figure on advantages/limitations for each approach? Key places where interdisciplinary opportunities (leverage)
\end{enumerate}


\section{Where to find other notes}
\verb|grephon/notes/2023SpringLucidEtc/Grephon_Lucid board notes - Google Docs.pdf|  is a file extracting the main take-homes from our Lucidboard work (December 2022 until sometime in early 2023). You can also see the Lucid board in PDF file: Grephon December 2022 brainstorm

See treeringnotes.pdf for our notes from reading papers in fall 2022. 


In August 2023, Lizzie reviewed this doc and the Lucid board (again) and a little the notes from last year, she pulled out the following:

Stuff to probably work into ms ...
\begin{itemize}
\item Biophysical constraints
\begin{itemize}
\item To what extent can the internal phenological program constrain C uptake/additional growth and influence the fate of C?
\item Do phenological shifts push the growing season into periods of low VPD? This should enhance growth if turgor is the limit, but might lead to more regional variation in response. Can we test this with treeNet data?
\end{itemize}
\item How much local adaptation is there in G x P?
\item Show PEP725 + ITRB overlay (w/ Rubén) and highlight potential of this data in understanding growth-phenology
\item  What data / experiments needs to be collected to determine this either way?
\item Close on why this is important to get right
\item Stuff we critically need to know
\begin{itemize}
\item When does growth happen for different species (early versus late season) and how flexible is this pattern?
\item How photoperiod matters (or doesn't and why folks get it wrong in dendro so much)
\item Better understanding of NSCs and phenology - are we seeing differences across wood anatomy or is it even more species specific or population specific
\item Differences in patterns across space/species/source populations?
\end{itemize}
\item It may not be a long-term stable strategy to try to adjust growth dramatically year-to-year, so should we really expect this correlation?
\item Cite Knott et al. 2022 `Phenological response to climate variation in a northernred oak plantation: Links to survival and productivity' -- no growth data, we really need more common garden studies
\end{itemize}

Especially cool ideas to do someday (but maybe not in this paper)
\begin{itemize}
\item What does the flux tower data tell us about the availability of photosynthate for growth? Can we use this to make assumptions about the sugars going elsewhere, like to heartwood?
\item Compare numbers: ring growth across elevations/across time/in dendro studies/dendrometers
\item What is the regional variation like in the growth phenology relationship? Are there places where growth is clearly increasing? If so, what is the climate like? Where would we expect growth to increase with a longer GS? Where would we not?
\item Statistical issues
\end{itemize}


\end{document}

\begin{enumerate}
\item
\end{enumerate}

\begin{itemize}
\item 
\end{itemize}