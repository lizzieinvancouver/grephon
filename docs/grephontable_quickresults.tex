\documentclass[11pt]{article}
\usepackage[top=1.00in, bottom=1.0in, left=1.1in, right=1.1in]{geometry}
\usepackage{Sweave}
\renewcommand{\baselinestretch}{1.1}
\usepackage{graphicx}
\usepackage{natbib}
\usepackage{amsmath}
\usepackage{hyperref}
% \externaldocument{}
\usepackage{parskip}


\usepackage{fancyhdr}
\pagestyle{fancy}
\fancyhead[LO]{May 2023}
\fancyhead[RO]{Grephon table}

\def\labelitemi{--}
\parindent=0pt

\begin{document}

\renewcommand{\refname}{\CHead{}}

\title{Grephon: What we learned from the papers}
\date{\today}
\author{Grephon group} 
\maketitle

\tableofcontents

\section{Quick results}

Most folks submitted their tables trying to digest papers on growing season length relates to growth -- thank you! We ended up with 35 rows of data across 21 papers. You can check out the merged file in the output folder \href{https://github.com/lizzieinvancouver/grephon/tree/main/analyses}{here}.

I did a quick review and then clean on some entries. You can look at the code (\verb|tablemergeclean.R|) in the \href{https://github.com/lizzieinvancouver/grephon/tree/main/analyses}{analyses folder}. Here's some info from that....

Most studies are temperate or boreal forests as best I can tell. Lots of \emph{Pinus, Abies, Betula, Fraxinus, Quercus, Fagus}.

Growth metrics were dominated by tree rings (annual cores):
\begin{Schunk}
\begin{Soutput}
                                 NDVI/LAI          annual core 
                   1                    2                   10 
  biomass/height/R:S dendrometer diameter    intra-annual core 
                   5                    2                    6 
               other       photosynthesis 
                   4                    5 
\end{Soutput}
\end{Schunk}

\newpage
Study types were dominated by tree rings (intra and inter-annual) but then more diverse:
\begin{Schunk}
\begin{Soutput}
                                         2 
continental scale obs phenology with model 
                                         1 
             ecosystem carbon budget model 
                                         1 
                                greenhouse 
                                         7 
                     greenhouse or chamber 
                                         1 
          intra-annual cores (xylogenesis) 
                                         2 
                                provenance 
                                         1 
                                 satellite 
                                         3 
shade and climate manipulation experiments 
                                         1 
            shade manipulation experiments 
                                         1 
                                 tree ring 
                                        15 
\end{Soutput}
\end{Schunk}
\newpage

In 6 papers and 10 rows of data, authors thought they found a relationship, but this varied with growth metric (you'll also see we're rather unsure about those intra-annual core studies):
\begin{Schunk}
\begin{Soutput}
                       no not mentioned unsure yes
                        0             0      0   0
  NDVI/LAI              0             2      0   0
  annual core           3             1      0   4
  biomass/height/R:S    1             0      0   3
  dendrometer diameter  2             0      0   0
  intra-annual core     2             0      4   0
  other                 2             0      0   2
  photosynthesis        0             1      0   1
\end{Soutput}
\end{Schunk}


And we're not so sure more than one row of data includes a growth x growing season relationship:
\begin{Schunk}
\begin{Soutput}
                  no unsure yes
  no            0  3      5   0
  not mentioned 0  0      0   0
  unsure        0  4      0   0
  yes           0  0      9   1
\end{Soutput}
\end{Schunk}

\section{Questions I think we need to answer before entering more data...}

\begin{enumerate}
\item Was this data entry doable? It was easy enough for me to clean quickly, but I did not hear how it went for others doing entry?
\item I and Ailene want some of our papers reviewed by someone else, do we want to just have everything checked twice?
\item Adjustments to data entry ...
\begin{enumerate}
\item What do we mean by `did authors think they found evidence?' ... I still struggled with this. Do we mean in whatever way they defined it? Do we want or have a column for GSL x growth (our version ... and what is our version? We could have a couple, see list below)? 
\item Are we separating out leaf from wood phenology studies enough?
\item How to enter xylogenesis studies?
\item I like the study level question, but I think it needs refining. Ailene added "Strideck et al 2022 study created tree ring chronologies (by merging tree rings across individuals within sites- a common practices in tree ring research) so I selected `Across sites' for study\_level. Might be worth a discussion as there may be other tree ring studies that use a similar approach." See below also ...
\end{enumerate}
\end{enumerate}

\newpage
\begin{Schunk}
\begin{Soutput}
                                                       1 
                                      across individuals 
                                                      10 
                                            across sites 
                                                       1 
                                across sites/populations 
                                                       9 
across sites/populations across years within individuals 
                                                       1 
                                          across species 
                                                       1 
                         across years within individuals 
                                                       7 
                 across years within individuals\302\240 
                                                       2 
                                      within individuals 
                                                       1 
 within individuals for < 1 year (April to October 2018) 
                                                       1 
                         within years within individuals 
                                                       1 
\end{Soutput}
\end{Schunk}


\begin{enumerate}
\item Can we write out the statements we want to make or line widths in a figure we want to define from this so we can make sure we're happy with the table?
\item What is our dream metric of GSL x growth?
\begin{enumerate}
\item GSL must be start to end for me -- NOT days growth $>$0 or such ... 
\item Does photosynthesis count as growth? What about the other random entries such as NDVI?
\end{enumerate}
\end{enumerate}

\section{Next steps}

\begin{enumerate}
\item Finalize the table again
\item Decide on how to assign additional reviews (re-reviewing) and assign!
\item Decide on aims to decide which papers we WOULD add
\item Do it ... 
\end{enumerate}

\end{document}

\section{References}
\bibliography{/Users/Lizzie/Documents/git/bibtex/LizzieMainMinimal}
\bibliographystyle{/Users/Lizzie/Documents/git/bibtex/styles/besjournals.bst}

Please find my attached table entries! I have several questions:
1) What to do when there are no species listed (e.g., Chen et al 2000 focused on "Canadian forests" with no mention at all of species. I put "NA" for now, but not sure that's the best entry...perhaps ">!"? or "no species listed" would be better?
2) Chen et al. found effects of disturbance (forest fire, insect-induced mortality, harvest) on growth (NPP)-does this count as a biophysical constraint (under endogenous factors)? 
3) The grephon_meta notes say "across years within individuals" includes " covers many tree ring studies (I think)" however, the Strideck et al 2022 study created tree ring chronologies (by merging tree rings across individuals within sites- a common practices in tree ring research) so I selected "Across sites" for study_level. Might be worth a discussion as there may be other tree ring studies that use a similar approach.
4) I would really like to have someone else read over the same paper (especially Zani) and see if they agree with my interpretation. IS this something we could do?

