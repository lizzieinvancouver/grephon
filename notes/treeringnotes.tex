\documentclass[11pt,letter]{article}
\usepackage[top=1.00in, bottom=1.0in, left=1.1in, right=1.1in]{geometry}
\renewcommand{\baselinestretch}{1.1}
\usepackage{graphicx}
\usepackage{natbib}
\usepackage{amsmath}
\usepackage{hyperref}

\def\labelitemi{--}
\parindent=0pt

\begin{document}
\bibliographystyle{/Users/Lizzie/Documents/EndnoteRelated/Bibtex/styles/besjournals}
\renewcommand{\refname}{\CHead{}}

{\bf Notes from fall 2022 reading group}\\
\begin{enumerate}
\item Week 1: Dow \emph{et al. 2022} and skim Gantois 2022
\begin{enumerate}
\item Curious about standardizing ... why they did it what way? Do results hold up with other methods?
\item What about autocorrelations in climate?
\item Heartwood! Roots!
\item What about VPDI? They should do analyses with that
\item They did not have much data in the end ... especially given they don't define how big an effect they expect to find or how big an effect they could detect (simulate data people!)
\item Why did Gantois exclude March from the spring?
\item Lots of good citations in Dow on other hypotheses/latitudinal variation to review!
\end{enumerate}
\item Week 2: Zohner \emph{et al. 2022} solstice pre-print
\begin{enumerate}
\item Paper diverges from Keenan work cited where longer seasons mean more carbon storage
\item Things that could explain the shift over time that looks to go with solstice and that earlier means you stop growing sooner (if you are  tree)
\begin{enumerate}
\item Drought correlates with earlier springs
\item NEW {\bf cool} hypothesis from Alana: $EOS_{10}$ is just a measure of herbivory ... so maybe this paper just means that earlier springs means earlier onset of 10\% herbivory 
\item Running out of nutrients?
\item Limited leaf lifespan
\item Specific to some species?
\item Something about radiation
\end{enumerate}
\item Kavya suggests they needed a daylength experiment to make the solstice argument convincing 
\item No changes over latitude seem weird 
\item Thermal optimum of leaf tissue can change up to 10C on the SAME tree each year, says Alana. 
\item See also the to do list below
\item Misc.
\begin{enumerate}
\item Some of the error reported is ODDLY small, suggesting the models are wrong somehow.
\item Needs better measures of senescence says Fredi -- check $A_{max}$ and other measurements and see how they correlate with MODIS
\item We're not sure that MODIS is not just measuring radiation also ... in which case part of the paper is circular but that does not explain the ground observational data
\item Radiation? ... constructing leaf tissue, increase productivity (not sure what I meant here says Lizzie looking at her notes later)
\end{enumerate}
\end{enumerate}
\item Week 3: Schofield \emph{et al. 2016} tree rings
\begin{enumerate}
\item This paper compares tree ring methods in a joint Bayesian models and shows that when you model altogether you get WILDLY different temperatures. 
\item Cat and Lizzie loved the math and paper, others found it dense and frustrating ... and we all agreed the figures are bad.
\item Ruben found discussions of uncertainty in the step-wise approach and in cross-dating important and cool.
\item Who has cited this paper?
\item Lizzie really enjoyed how they used the models to show support for classic approach and less support for RCS and how much a slightly more realistic biology sucked up variation and climate stuff. 
\item New paper idea from JHRL: Write an ecological paper with a joint model and show that you get something NEW out of it 
\end{enumerate}
\item Week 4:  Tumajer \emph{et al. 2020} the VS lite model, which gives your start of wood growth and end of wood growth from annual tree ring data via a process-based model ... See Fig. 1 in the supp for how well the model fits to real xylogenesis data
\begin{enumerate}
\item About the paper
\begin{enumerate}
\item No one is sure about the model -- what's with all the tuning (what does tuning mean)? Did they tune to sites or not? We need to understand the validation stuff better, and it would be cool to know how their model works if you run white noise through it versus a model where you generate data from the VS model and then add noise or such. 
\item Ruben asked -- is this model possible? Do we believe you could build it? Alana says if these things do work, then where in the tree you measure this would matter...
\end{enumerate}
\item Other things we discussed ...
\begin{enumerate}
\item Does growth change with longer seasons? Depends on whether you're talking about ind. species versus communities? Communities should get longer, but we're not sure on species ... but most of the authors seem to be assuming all species act the same. 
\item Earlier leaf phenology does not mean more wood, because leaves go into sugar not obviously wood... says Alana. 
\item Physic-focused vs. observational/statistical groups of physiologists ... and who works on tree rings? (The latter.)
\item What do we know about xylogenesis? Does winter temperature matter? It could says Alana, esp. with winter embolusism but she's not sure of anyone working on this. 
\end{enumerate}
\end{enumerate}
\item Week 5:  Soolananayakanahally \emph{et al.} 2013 (notes include week 5 discussion and discussion from 
\begin{enumerate}
\item Vocab: LS - leaf senescence, HGC -- height growth cessation, and (lammas is a bud you make after you've made your other buds and (I think...?) see wiki
\item Summarise: BB is sensitive to warming as we know, but HGC also responds in opposite direction (earlier with earlier springs) -- they have bud flush versus bud-set and leaf senenscence... HGC variation was higher in Vancouver (and lots of variation)
% They show earlier leafout with earlier bud-set, but it wasn't clear who grew more -- because R:S seemed to change (more roots in Vancouver) and the Vancouver site had more Lammas (they grew a lot more random leaves). 
\item HGC ... happens before LS. HGC -- their evidence that it reversed?
\item Is this whole photoperiod thing a good idea?
\begin{enumerate}
\item Why is photoperiod so adaptive? It seems tough to use as a cue for budset (aka mostly stopping growing)?
\item Is lammas growth a way to get in some more growth despite the photoperiod/budset cues. 
\item Why do plants use photoperiod to stop growth?
\end{enumerate}
\item Design issues ... Is this potentially maladaptive because the common garden was so far south (and one out of range)? Do they have any up higher?
\item More physiology info please! How much leaf area did lammas make up? Is lammas common across years for this species? Ray storage -- did they measure that?
% says Alana
\item Is it really latitude? Did they look at other metrics to predict variation instead of latitude? MAT or climate variability?
\item (From Cat): They mention that length of night is what matters (so maybe wavelength during the day doesn't matter so much).
\item If someone had a lot of time they could play around with stop and starting growth in trees in one season (in a greenhouse or such) and then measure tree rings the next year. They could look across species also to see how they vary. Ruben had a cool common garden study in Vancouver where they measured tree rings ... 
\item Is this species diffuse or ring porous?  ... Lizzie's old database (\url{https://www.wood-database.com/hardwoods/}) may not have it but the USDA (\url{https://plants.usda.gov/DocumentLibrary/plantguide/pdf/pg_pobat.pdf} says ... diffuse!
\item Should we do more tree ring measurements in common gardens (possibly related `Cold adaptation recorded in tree rings highlights risks associated with climate change and assisted migration' see \url{https://www.nature.com/articles/s41467-018-04039-5})
\item What is the minimum amount of physiology that Alana wants? NSC weekly would be useful (and something else I missed)
% Do they have biomass as a f(x) of GCP?
% Did they count or otherwise measure bud #?
% Do they see lag effects -- where earlier springs lead to bigger rings in the next year? 
% Zohner et al. -- climate might be variability 
% Which species are indeterminate and determinate? And do you seem the same shift in HGC with determinate species?
\end{enumerate}
\item Week 6:  What do we wish we knew?
\begin{enumerate}
\item Iseful exercise: What papers we wish exist? Where are the gaps. 
\item To get to this Lizzie will assign us each a paper we already read and we need to answer -- What did they measure? What did they find? What is their theory/mechanisms? What are the gaps you see? Are any of those gaps filled by other papers or data you know of that would be easy to get? 
\end{enumerate}
\end{enumerate}


{\bf What do we want to do next?}\\

There are sort of two big possible ways to go -- think broadly about what could drive trends (or lack thereof) between season length and growth (and ideally jot down a concept paper) \emph{or} or dive in on some data and related analyses we want to do. These sort of overlap ... if we start with the latter we likely need to return to the former.

\begin{enumerate}
\item Where is this all going and how do we get there? 
\begin{enumerate}
\item Are we talking about community or species-level trends?
\item If you had all the data, and all the models are correct; what is most urgent? What is most important to know? 
\item If we can link phenology to cambial growth; could we understand how this shifts with climate change and thus carbon sequestration? [Relating to above, can we make that link?]
\item Scroll to last page maybe -- see `Lizzie’s earlier notes'
\end{enumerate}
\item Can we link leaf phenology to tree ring growth? ... with some data... 
\begin{enumerate}
\item Maybe through common garden data? (Aitken lab etc.) ... Do we have tree ring, bud-set, start of season data? Someone must have done dendro?
\item Can we get good high-resolution satellite data to look at EOS by species (we asked Kavya)? Some cool 'cubesat' stuff ... Cat added: GEDI is now 30m x 30m resolution % Whatever happened with O'Keefe phenology? Pederson is measuring a lot of stuff on these trees. 
\end{enumerate}
\end{enumerate}


{\bf Paper ideas}\\
Interesting papers to read:
\begin{enumerate}
\item \url{https://besjournals.onlinelibrary.wiley.com/doi/full/10.1111/1365-2745.13464} (Some of Jucker last papers, super interesting approach, used something similar to the RCS but then use those to calculate the over/under production. Is not phenology, but shows I think a bit better how to use dendro in a better way)
\item VS-lite models: this is a way to build physiological constrains (very similar to what they did in Schofield paper with the thresholds) to decompose tree ring into intra-annual patterns and then model it forward. It's really interesting and loads of paper but it really feels like it shouldn't work. I would be curious to dig in and get people's feedback, specially of those most experience in these. Definitively has more links to phenology if we consider it robust:
\begin{enumerate}
\item "Original" paper: Tolwinski-Ward: \url{https://link.springer.com/article/10.1007/s00382-010-0945-5}
\item Example of application: \url{https://onlinelibrary.wiley.com/doi/full/10.1111/geb.13377}
\end{enumerate}
\item Aloni R (2022) How the Three Organ-Produced Signals: Auxin, Cytokinin and Gibberellin, Induce and Regulate Wood Formation and Adaptation. In: Auxins, Cytokinins and Gibberellins Signaling in Plants. T Aftab (Ed), Springer Nature, Cham, Switzerland. % Alana adds: This paper is slightly off topic in that it isn't about phenology at all, but I think it might help us put together ideas about WHY things like bud break timing influence wood devo.
\item Decoupled leaf-wood phenology in two pine species from contrasting climates: Longer growing seasons do not mean more radial growth
\item A photoperiod-budset paper
\item Papers on VPDI and heartwood? (sink limitation)
\item Any other papers on phenology and tree growth??!!
\end{enumerate}



From Ruben:

Groups/people doing 'modern ecology' with tree-rings to keep an eye on:
\begin{enumerate}
\item Margaret Evans. Arizona. \url{https://scholar.google.com/citations?hl=en&user=IGG0ZKQAAAAJ&view_op=list_works&sortby=pubdate}
\item Charlotte Grossiord. Lausanne. \url{https://scholar.google.com/citations?hl=en&user=RsHW0OsAAAAJ&view_op=list_works&sortby=pubdate}
\item Tommasso Jucker. Bristol. \url{https://scholar.google.com/citations?hl=en&user=s0x7E5wAAAAJ&view_op=list_works&sortby=pubdate}
\item Valerie Trouet. Belgium. \url{https://scholar.google.com/citations?hl=en&user=-hF1HN8AAAAJ&view_op=list_works&sortby=pubdate}
\item Loic D'Orangeville. New Brumswick. \url{https://scholar.google.com/citations?hl=en&user=CwBKApgAAAAJ&view_op=list_works&sortby=pubdate}
\item Dario Martin-Benito. Madrid. \url{https://scholar.google.com/citations?hl=en&user=Qiooe3EAAAAJ&view_op=list_works&sortby=pubdate}
\end{enumerate}

Zhao et al. 2018 is the paper where we commented about this project together with Shoudong of characterizing the bias of the ITRDB and where we proposed some ideas on how to tackle it. It is the base for the ERC proposal but not much else went with it afterwards (and the database went again back to get filled with problems, but well... we tried)\\

{\bf To do items ... maybe?}
\begin{enumerate}
\item Email Keenan to see his perspectives on Zohner preprint; also ask Norby? % Richardson, A. D. et al. Climate change, phenology, and phenological control of vegetation feedbacks to the climate system. Agric. For. Meteorol. 169, 156–173 (2013).
 ... Keenan, T. F. et al. Net carbon uptake has increased through warming-induced changes in temperate forest phenology. Nat. Clim. Chang. 4, 598–604 (2014). Do they also use MODIS GPP?
\item Check refs of earlier spring = later EOS in Zohner preprint and COMPARE the papers
\item For the Zohner preprint: Check if budbset is earlier in years with early springs (from common gardens with multiple years of data)
\item Review refs in Pederson for latitude and other things ... here's all the ones Lizzie highlighted (last 3 are latitude).
\begin{enumerate}
\item Ahlstrom, A., Schurgers, G., Arneth, A. \& Smith, B. Robustness and uncertainty in terrestrial ecosystem carbon response to CMIP5 climate change projections. Environ. Res. Lett. 7, 044008 (2012).
\item Zweifel, R. et al. Why trees grow at night. New Phytol. 231, 2174–2185 (2021).
\item Tumajer, J., Scharnweber, T., Smiljanic, M. \& Wilmking, M. Limitation by vapour pressure
deficit shapes different intra-annual growth patterns of diffuse- and ring-porous
temperate broadleaves. New Phytol. 233, 2429–2441 (2022).
\item Cabon, A. et al. Cross-biome synthesis of source versus sink limits to tree growth.
Science 376, 758–761 (2022).
\item D'Orangeville, L. et al. Drought timing and local climate determine the sensitivity of
eastern temperate forests to drought. Glob. Chang. Biol. 24, 2339–2351 (2018).
\item Helcoski, R. et al. Growing season moisture drives interannual variation in woody productivity of a temperate deciduous forest. New Phytol. 223, 1204–1216 (2019).
\item Anderson-Teixeira, K. J. et al. Joint effects of climate, tree size, and year on annual tree growth derived from tree-ring records of ten globally distributed forests. Glob. Chang. Biol. 28, 245–266 (2022).
\item Banbury Morgan, R. et al. Global patterns of forest autotrophic carbon fluxes. Glob. Chang. Biol. 27, 2840–2855 (2021).
\item Churkina, G., Schimel, D., Braswell, B. H. \& Xiao, X. Spatial analysis of growing season length control over net ecosystem exchange. Glob. Chang. Biol. 11, 1777–1787 (2005).
\end{enumerate}
\item Check who has cited Schofield et al. 2016 since it was published!
\item VS lite model: How to know what the null trends or findings are (Tumajer paper discussion notes above: would be cool to know how their model works if you run white noise through it versus a model where you generate data from the VS model and then add noise or such). 
\end{enumerate}


\newpage
{\bf Lizzie's earlier notes below}\\

\emph{Background:} As springs shift growing seasons lengthen and plants are expected to grow longer. Especially trees, but tree rings suggest growth may not be increasing with earlier seasons in temperate zones. \\

{\bf Hypotheses:}
\begin{enumerate}
\item Statistical -- Non-stationarity in temperature (climate) data may make accurately estimating phenological change and tree growth change accurately difficult. Check this early and often. 
\item Climate correlations -- warmer springs may be associated with factors that reduce plant growth such as drought (and/or did someone write something about winter chilling?). Relates to climate hazards work. 
\item Ecology -- shifting competitive landscapes (or something else?) 
\item Evolution -- It may not be a long-term stable strategy to try to adjust growth dramatically year-to-year, so should we really expect this correlation? If this is true, you predict:
\begin{enumerate}
\item Latitudinal variation in length of growing season and tree growth. (May connect back to Ailene's Putnam -- predicts species from warmer provenances would better exploit longer growing seasons?)  ... also these papers cited in Dow 2022: 44. Anderson-Teixeira, K. J. et al. Joint effects of climate, tree size, and year on annual tree growth derived from tree-ring records of ten globally distributed forests. Glob. Chang. Biol. 28, 245–266 (2022). 45. Banbury Morgan, R. et al. Global patterns of forest autotrophic carbon fluxes. Glob. Chang. Biol. 27, 2840–2855 (2021). 46. Churkina, G., Schimel, D., Braswell, B. H. \& Xiao, X. Spatial analysis of growing season length control over net ecosystem exchange. Glob. Chang. Biol. 11, 1777–1787 (2005).
\item Species diversity: species should vary in how much they try to take advantage of interannual variation in climate (likely early-active species show the highest correlation? Again, Ailene's Putnam focused on this.)
\end{enumerate}
\end{enumerate}

\bibliography{..//refs/micro}

\end{document}