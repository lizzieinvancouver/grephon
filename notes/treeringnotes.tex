\documentclass[11pt,letter]{article}
\usepackage[top=1.00in, bottom=1.0in, left=1.1in, right=1.1in]{geometry}
\renewcommand{\baselinestretch}{1.1}
\usepackage{graphicx}
\usepackage{natbib}
\usepackage{amsmath}
\usepackage{hyperref}

\def\labelitemi{--}
\parindent=0pt

\begin{document}
\bibliographystyle{/Users/Lizzie/Documents/EndnoteRelated/Bibtex/styles/besjournals}
\renewcommand{\refname}{\CHead{}}

{\bf Notes from fall 2022 reading group}\\
\begin{enumerate}
\item Week 1: Dow \emph{et al. 2022} and skim Gantois 2022
\begin{enumerate}
\item Curious about standardizing ... why they did it what way? Do results hold up with other methods?
\item What about autocorrelations in climate?
\item Heartwood! Roots!
\item What about VPDI? They should do analyses with that
\item They did not have much data in the end ... especially given they don't define how big an effect they expect to find or how big an effect they could detect (simulate data people!)
\item Why did Gantois exclude March from the spring?
\item Lots of good citations in Dow on other hypotheses/latitudinal variation to review!
\end{enumerate}
\item Week 2: Zohner \emph{et al. 2022} solstice pre-print
\begin{enumerate}
\item Paper diverges from Keenan work cited where longer seasons mean more carbon storage
\item Things that could explain the shift over time that looks to go with solstice and that earlier means you stop growing sooner (if you are  tree)
\begin{enumerate}
\item Drought correlates with earlier springs
\item NEW {\bf cool} hypothesis from Alana: $EOS_{10}$ is just a measure of herbivory ... so maybe this paper just means that earlier springs means earlier onset of 10\% herbivory 
\item Running out of nutrients?
\item Limited leaf lifespan
\item Specific to some species?
\item Something about radiation
\item{enumerate}
\item Kavya suggests they needed a daylength experiment to make the solstice argument convincing 
\item No changes over latitude seem weird 
\item Thermal optimum of leaf tissue can change up to 10C on the SAME tree each year, says Alana. 
\item See also the to do list below
\item Misc.
\begin{enumerate}
\item Some of the error reported is ODDLY small, suggesting the models are wrong somehow.
\item Needs better measures of senescence says Fredi -- check $A_{max}$ and other measurements and see how they correlate with MODIS
\item We're not sure that MODIS is not just measuring radiation also ... in which case part of the paper is circular but that does not explain the ground observational data
\item Radiation? ... constructing leaf tissue, increase productivity (not sure what I meant here says Lizzie looking at her notes later)
\item{enumerate}
\end{enumerate}
\end{enumerate}
\item Week 3: Schofield \emph{et al. 2016} tree rings
\begin{enumerate}
\item This paper compares tree ring methods in a joint Bayesian models and shows that when you model altogether you get WILDLY different temperatures. 
\item Cat and Lizzie loved the math and paper, others found it dense and frustrating ... and we all agreed the figures are bad.
\item Ruben found discussions of uncertainty in the step-wise approach and in cross-dating important and cool.
\item Who has cited this paper?
\item Lizzie really enjoyed how they used the models to show support for classic approach and less support for RCS and how much a slightly more realistic biology sucked up variation and climate stuff. 
\item New paper idea from JHRL: Write an ecological paper with a joint model and show that you get something NEW out of it 
\end{enumerate}
\end{enumerate}


{\bf Paper ideas}\\
Interesting papers to read:
\begin{enumerate}
\item \url{https://besjournals.onlinelibrary.wiley.com/doi/full/10.1111/1365-2745.13464} (Some of Jucker last papers, super interesting approach, used something similar to the RCS but then use those to calculate the over/under production. Is not phenology, but shows I think a bit better how to use dendro in a better way)
\item VS-lite models: this is a way to build physiological constrains (very similar to what they did in Schofield paper with the thresholds) to decompose tree ring into intra-annual patterns and then model it forward. It's really interesting and loads of paper but it really feels like it shouldn't work. I would be curious to dig in and get people's feedback, specially of those most experience in these. Definitively has more links to phenology if we consider it robust:
\begin{enumerate}
\item "Original" paper: Tolwinski-Ward: \url{https://link.springer.com/article/10.1007/s00382-010-0945-5}
\item Example of application: \url{https://onlinelibrary.wiley.com/doi/full/10.1111/geb.13377}
\end{enumerate}
\item Aloni R (2022) How the Three Organ-Produced Signals: Auxin, Cytokinin and Gibberellin, Induce and Regulate Wood Formation and Adaptation. In: Auxins, Cytokinins and Gibberellins Signaling in Plants. T Aftab (Ed), Springer Nature, Cham, Switzerland. % Alana adds: This paper is slightly off topic in that it isn't about phenology at all, but I think it might help us put together ideas about WHY things like bud break timing influence wood devo.
\item Decoupled leaf-wood phenology in two pine species from contrasting climates: Longer growing seasons do not mean more radial growth
\item A photoperiod-budset paper
\item Papers on VPDI and heartwood? (sink limitation)
\item Any other papers on phenology and tree growth??!!
\end{enumerate}



From Ruben:

Groups/people doing 'modern ecology' with tree-rings to keep an eye on:
\begin{enumerate}
\item Margaret Evans. Arizona. \url{https://scholar.google.com/citations?hl=en&user=IGG0ZKQAAAAJ&view_op=list_works&sortby=pubdate}
\item Charlotte Grossiord. Lausanne. \url{https://scholar.google.com/citations?hl=en&user=RsHW0OsAAAAJ&view_op=list_works&sortby=pubdate}
\item Tommasso Jucker. Bristol. \url{https://scholar.google.com/citations?hl=en&user=s0x7E5wAAAAJ&view_op=list_works&sortby=pubdate}
\item Valerie Trouet. Belgium. \url{https://scholar.google.com/citations?hl=en&user=-hF1HN8AAAAJ&view_op=list_works&sortby=pubdate}
\item Loic D'Orangeville. New Brumswick. \url{https://scholar.google.com/citations?hl=en&user=CwBKApgAAAAJ&view_op=list_works&sortby=pubdate}
\item Dario Martin-Benito. Madrid. \url{https://scholar.google.com/citations?hl=en&user=Qiooe3EAAAAJ&view_op=list_works&sortby=pubdate}
\end{enumerate}

Zhao et al. 2018 is the paper where we commented about this project together with Shoudong of characterizing the bias of the ITRDB and where we proposed some ideas on how to tackle it. It is the base for the ERC proposal but not much else went with it afterwards (and the database went again back to get filled with problems, but well... we tried)

{\bf To do items ... maybe?}
\begin{enumerate}
\item Email Keenan to see his perspectives on Zohner preprint; also ask Norby? % Richardson, A. D. et al. Climate change, phenology, and phenological control of vegetation feedbacks to the climate system. Agric. For. Meteorol. 169, 156–173 (2013).
 ... Keenan, T. F. et al. Net carbon uptake has increased through warming-induced changes in temperate forest phenology. Nat. Clim. Chang. 4, 598–604 (2014). Do they also use MODIS GPP?
\item Check refs of earlier spring = later EOS in Zohner preprint and COMPARE the papers
\item For the Zohner preprint: Check if budbset is earlier in years with early springs (from common gardens with multiple years of data)
\item Review refs in Pederson for latitude and other things ... here's all the ones Lizzie highlighted (last 3 are latitude).
\begin{enumerate}
\item Ahlstrom, A., Schurgers, G., Arneth, A. \& Smith, B. Robustness and uncertainty in terrestrial ecosystem carbon response to CMIP5 climate change projections. Environ. Res. Lett. 7, 044008 (2012).
\item Zweifel, R. et al. Why trees grow at night. New Phytol. 231, 2174–2185 (2021).
\item Tumajer, J., Scharnweber, T., Smiljanic, M. \& Wilmking, M. Limitation by vapour pressure
deficit shapes different intra-annual growth patterns of diffuse- and ring-porous
temperate broadleaves. New Phytol. 233, 2429–2441 (2022).
\item Cabon, A. et al. Cross-biome synthesis of source versus sink limits to tree growth.
Science 376, 758–761 (2022).
\item D'Orangeville, L. et al. Drought timing and local climate determine the sensitivity of
eastern temperate forests to drought. Glob. Chang. Biol. 24, 2339–2351 (2018).
\item Helcoski, R. et al. Growing season moisture drives interannual variation in woody productivity of a temperate deciduous forest. New Phytol. 223, 1204–1216 (2019).
\item Anderson-Teixeira, K. J. et al. Joint effects of climate, tree size, and year on annual tree growth derived from tree-ring records of ten globally distributed forests. Glob. Chang. Biol. 28, 245–266 (2022).
\item Banbury Morgan, R. et al. Global patterns of forest autotrophic carbon fluxes. Glob. Chang. Biol. 27, 2840–2855 (2021).
\item Churkina, G., Schimel, D., Braswell, B. H. \& Xiao, X. Spatial analysis of growing season length control over net ecosystem exchange. Glob. Chang. Biol. 11, 1777–1787 (2005).
\end{enumerate}
\item Check who has cited Schofield et al. 2016 since it was published!
\end{enumerate}


\newpage
{\bf Lizzie's earlier notes below}\\

\emph{Background:} As springs shift growing seasons lengthen and plants are expected to grow longer. Especially trees, but tree rings suggest growth may not be increasing with earlier seasons in temperate zones. \\

{\bf Hypotheses:}
\begin{enumerate}
\item Statistical -- Non-stationarity in temperature (climate) data may make accurately estimating phenological change and tree growth change accurately difficult. Check this early and often. 
\item Climate correlations -- warmer springs may be associated with factors that reduce plant growth such as drought (and/or did someone write something about winter chilling?). Relates to climate hazards work. 
\item Ecology -- shifting competitive landscapes (or something else?) 
\item Evolution -- It may not be a long-term stable strategy to try to adjust growth dramatically year-to-year, so should we really expect this correlation? If this is true, you predict:
\begin{enumerate}
\item Latitudinal variation in length of growing season and tree growth. (May connect back to Ailene's Putnam -- predicts species from warmer provenances would better exploit longer growing seasons?)  ... also these papers cited in Dow 2022: 44. Anderson-Teixeira, K. J. et al. Joint effects of climate, tree size, and year on annual tree growth derived from tree-ring records of ten globally distributed forests. Glob. Chang. Biol. 28, 245–266 (2022). 45. Banbury Morgan, R. et al. Global patterns of forest autotrophic carbon fluxes. Glob. Chang. Biol. 27, 2840–2855 (2021). 46. Churkina, G., Schimel, D., Braswell, B. H. \& Xiao, X. Spatial analysis of growing season length control over net ecosystem exchange. Glob. Chang. Biol. 11, 1777–1787 (2005).
\item Species diversity: species should vary in how much they try to take advantage of interannual variation in climate (likely early-active species show the highest correlation? Again, Ailene's Putnam focused on this.)
\end{enumerate}
\end{enumerate}

\bibliography{..//refs/micro}

\end{document}