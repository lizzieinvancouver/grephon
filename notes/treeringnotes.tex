\documentclass[11pt,letter]{article}
\usepackage[top=1.00in, bottom=1.0in, left=1.1in, right=1.1in]{geometry}
\renewcommand{\baselinestretch}{1.1}
\usepackage{graphicx}
\usepackage{natbib}
\usepackage{amsmath}

\def\labelitemi{--}
\parindent=0pt

\begin{document}
\bibliographystyle{/Users/Lizzie/Documents/EndnoteRelated/Bibtex/styles/besjournals}
\renewcommand{\refname}{\CHead{}}

{\bf Notes from fall 2022 reading group}\\
\begin{enumerate}
\item Week 1: Dow \emph{et al. 2022} and skim Gantois 2022
\begin{enumerate}
\item Curious about standardizing ... why they did it what way? Do results hold up with other methods?
\item What about autocorrelations in climate?
\item Heartwood! Roots!
\item What about VPDI? They should do analyses with that
\item They did not have much data in the end ... especially given they don't define how big an effect they expect to find or how big an effect they could detect (simulate data people!)
\item Why did Gantois exclude March from the spring?
\item Lots of good citations in Dow on other hypotheses/latitudinal variation to review!
\end{enumerate}
\item Week 1: Zohner \emph{et al. 2022} solstice pre-print
\end{enumerate}

\newpage
{\bf Lizzie's earlier notes below}\\


\emph{Background:} As springs shift growing seasons lengthen and plants are expected to grow longer. Especially trees, but tree rings suggest growth may not be increasing with earlier seasons in temperate zones. \\

{\bf Hypotheses:}
\begin{enumerate}
\item Statistical -- Non-stationarity in temperature (climate) data may make accurately estimating phenological change and tree growth change accurately difficult. Check this early and often. 
\item Climate correlations -- warmer springs may be associated with factors that reduce plant growth such as drought (and/or did someone write something about winter chilling?). Relates to climate hazards work. 
\item Ecology -- shifting competitive landscapes (or something else?) 
\item Evolution -- It may not be a long-term stable strategy to try to adjust growth dramatically year-to-year, so should we really expect this correlation? If this is true, you predict:
\begin{enumerate}
\item Latitudinal variation in length of growing season and tree growth. (May connect back to Ailene's Putnam -- predicts species from warmer provenances would better exploit longer growing seasons?)  ... also these papers cited in Dow 2022: 44. Anderson-Teixeira, K. J. et al. Joint effects of climate, tree size, and year on annual tree growth derived from tree-ring records of ten globally distributed forests. Glob. Chang. Biol. 28, 245–266 (2022). 45. Banbury Morgan, R. et al. Global patterns of forest autotrophic carbon fluxes. Glob. Chang. Biol. 27, 2840–2855 (2021). 46. Churkina, G., Schimel, D., Braswell, B. H. \& Xiao, X. Spatial analysis of growing season length control over net ecosystem exchange. Glob. Chang. Biol. 11, 1777–1787 (2005).
\item Species diversity: species should vary in how much they try to take advantage of interannual variation in climate (likely early-active species show the highest correlation? Again, Ailene's Putnam focused on this.)
\end{enumerate}
\end{enumerate}

\bibliography{..//refs/micro}

\end{document}