\documentclass[11pt,letter]{article}
\usepackage[top=1.00in, bottom=1.0in, left=1.1in, right=1.1in]{geometry}
\renewcommand{\baselinestretch}{1.1}
\usepackage{graphicx}
\usepackage{natbib}
\usepackage{amsmath}
\usepackage{hyperref}


\def\labelitemi{--}
\parindent=0pt

\begin{document}
\bibliographystyle{/Users/Lizzie/Documents/EndnoteRelated/Bibtex/styles/besjournals}
\renewcommand{\refname}{\CHead{}}

{\bf \large Notes from fall 2022 reading group}\\

{\b Week 8: Paper round robin}\\


\begin{enumerate}
\item (Frederik): My brief summary for the paper D'Orangeville, L. et al. (2018):
Trees across eastern Nordamerica are most sensitive to drought around the summer solstice during peak radial growth activity. Total growth was therefore mainly determined by the conditions during this time (Mai-July) and barely at the end of the GS (growing season).
\end{enumerate}



{\bf Week 8: What should we do?}\\

\emph{Come up with at least two ideas of what you think this group should work on next (examples include but are not limited to: a perspectives paper on XX topic, a synthetic review of YY, the most critical experiment for this topic that a MSc student could do this spring).}

\begin{enumerate}
\item Alana
\begin{enumerate}
\item I am leaning towards a concept paper that covers the ways in which phenology might influence growth and how that might vary depending on the phenological stage we are discussing. 
\item Alternatively, something a bit similar that reviews the mechanisms behind each major stage and discusses how they might influence growth.
% I am not convinced that we know enough to get far, but maybe we can outline experiments to test each idea we present?
\end{enumerate}
\item Frederik
\begin{enumerate}
\item (D'Orangeville 2018 paper) I think there is an important lesson (at least for me) that the authors discuss later: Potential effects of drought (but probably also stimuli) towards the end of the GS may be seen only in the next year! In some species shoots are already prebuilt in late summer for next year and reduced water as well as NSC reserves might influence performance in the next year. I think this is also an interesting gap for a future experiment:
\begin{enumerate}
\item 3 Treatments:
\begin{enumerate}
\item Control (normal growing season length with perfect conditions)
\item GS extended (potentially to split up even in spring and autumn extension)
\item GS shortened (drought or cool temp) in autumn (or again to split up an additional treatment in spring)
\end{enumerate}
\item After this treatment during the first GS all saplings experience another GS under the same control conditions. Then big harvest...
\end{enumerate}
\item I could think also of another experiment that focuses more on the faith of C, because it is still unclear to me when C is mainly taken up and where it ends up greatly influencing its residence time: Perhaps a CO$_2$ labeling experiment with different labeling occasions within the GS to evaluate
\begin{enumerate}
\item  How efficiently was the C fixed and
\item  Where is the fixed sugar incorporated (structural growth in different organs, starch reserves, root exudates…)
\end{enumerate}
\end{enumerate}
\item Kavya
\begin{enumerate}
\item Concept paper on climate, phenology, growth – what we know, and what are things the research community should work on perhaps emphasizing the need more more integration of physiology and phenology (and maybe molecular bio)
\item Get all common garden data possible to see if we could get some of the questions answered related to phenology (are there datasets already of phenology studies in common gardens that haven't specifically looked at growth but could be used to do so) and growth (could we core some of these trees to get at overall and interannual growth, at least for radial growth) – what data is present, what is missing, can we supplement missing data, what species can we look at?
\item Definitely also interested in seeing where the carbon is going (like Fredi) in years of longer or earlier growing season if not to radial growth although it would be hard for us to do this one
\item Flux tower work inspired by and complementary to work that has been done already perhaps: When does growth actually happen within the growing season - we could use flux tower data (though I have no idea how to even begin looking at this kind of data) to look at when in the year carbon drawdown is happening – if earlier spring phenology means earlier carbon drawdown then even if there isn’t tree radial growth, net carbon uptake is a good thing, maybe not for sequestration but short term drawdown? It would be cool to see if precipitation and vpd limitations later in the growing season lead to lower C uptake
\end{enumerate}
\item Janneke
\begin{enumerate}
\item Outline a review / synthesis / future directions paper that specifically focuses on when and how changes in phenology influence tree growth, and where that carbon goes. From the papers I've read I think there are few that integrate insights and measurement techniques from all subfields (phenology, tree ring, ecosystem / climate change), and a surprising number of unknowns. 
\item I think while we are doing this we should outline the perfect observational study and experiments to illustrate key unknowns. This will be good for the paper (more interesting), but also will help us decide whether any of us (in whatever combination of people) will want to start such a study / observational data set / analysis.    
\end{enumerate}
\item Lizzie
\begin{enumerate}
\item Important question -- dendro people who do temperature reconstructions with ring width, do they use the same year? (Meaning -- their whole field correlates the year with the climate??!!)
\item Review paper integrating community, ecology, dendro and physio perspective (so we skip the flux/remote sensing people) ... Title is something like `Chasms in tree growth' and I am working on a theme song.
\item Figure out the ideal 1-2 year greenhouse experiments to do to nail down variation across species in phenology x growth (x climate?) relationships ....
\item Using elevational tree ring data look both at elevational trends in ring width and trends over time (bracket variation possible and compare to Dow et al. or such? Basically get some numbers here ...).
\item Compare elevational trends in ring with with provenance trends (from common gardens)?
\end{enumerate}
\end{enumerate}




\end{document}



\bibliography{..//refs/micro}
