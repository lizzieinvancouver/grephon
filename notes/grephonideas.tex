\documentclass[11pt,letter]{article}
\usepackage[top=1.00in, bottom=1.0in, left=1.1in, right=1.1in]{geometry}
\renewcommand{\baselinestretch}{1.1}
\usepackage{graphicx}
\usepackage{natbib}
\usepackage{amsmath}
\usepackage{hyperref}


\def\labelitemi{--}
\parindent=0pt

\begin{document}
\bibliographystyle{/Users/Lizzie/Documents/EndnoteRelated/Bibtex/styles/besjournals}
\renewcommand{\refname}{\CHead{}}

{\bf \large Notes from fall 2022 reading group}\\


{\bf Week 8: What should we do?}\\

\emph{Notes during the meeting}\\

{\bf Take-home:} We decided to focus on a review paper addressing: Do longer growing seasons lead to more growth? (Not clear how much we get into carbon storage) ... see further notes below on what might be in this. We felt many of data/experiments could also fit in if we do it right (provide a road map for what we know and what we critically need to do next).\\

{\bf When? When will this happen?} We plan to start in mid-January and aim for a draft done in July (in between we need a first-author to stand up and lead the way. \\

{\bf What you need to do now!} Fill out the when2meet and the brainstorming board (set up a Lucid account) by {\bf 21 December 2022}. If you need more time for the brainstorming, just tell me when to expect it, but please do the when2meet ASAP. \\

We grouped ideas during the meeting:
\begin{enumerate}
\item Review paper (but with an opinion flair to it)
\begin{enumerate}
\item Relationships between growing season length, growth, and maybe carbon storage (Kavya and Ailene seemed interested... maybe also Cat?)
\item Do longer growing seasons lead to more growth? (Similar idea by Ailene, Ruben, Janneke)
\begin{enumerate}
\item Compare existing data and paper
\item compare spatial and temporal trends
\item Make some flow diagrams of how this works, look at each connection and ask if we have support? If so, how much? If not, what do we need to do to test it?
\item Answer: What experiments/data do we really need?
\end{enumerate}
\item Mini-version of previous, but with no data added
\item C sequestration and growth -- Alana, Cat and Ailene might be into this (the rest of us less so, but support them making this topic happen .... could also be a box in our bigger paper) ... 
\end{enumerate}
\item Concept/perspectives paper
\begin{enumerate}
\item NCS and feedbacks between climate and vegetation (Ailene and Cat?)
\item When does phenology matter to growth and for what phenological stages? (Alana interested, we think Frederik may be also)
\item Mechanisms of each phenological stage and how they relate to growth
\end{enumerate}
\item Data to get ...
\begin{enumerate}
\item Core all the common gardens with phenology data! (Or do cutting experiments adds Lizzie later)
\item Maybe do something with flux tower data?
\item PEP725 and ITRDB data -- get them together! (use as figure in paper?)
\item Compare elevational trends with temporal trends in ring width
\item Compare dendrometer values with tree ring values and dendro folks' values with community folks' who do dendro (like Janneke and Ailene) values
\end{enumerate}
\item Experiments to do ... 
\begin{enumerate}
\item C labeling to see where C goes with longer seasons
\item Greenhouse/chamber study with early/regular/short growing seasons (and maybe drought) then grow plants another year and measure rings. 
\end{enumerate}
\end{enumerate}

\vspace{5ex}\\
\emph{Notes from before the meeting started ... }\\

\emph{Come up with at least two ideas of what you think this group should work on next (examples include but are not limited to: a perspectives paper on XX topic, a synthetic review of YY, the most critical experiment for this topic that a MSc student could do this spring).}

\begin{enumerate}
\item Alana
\begin{enumerate}
\item I am leaning towards a concept paper that covers the ways in which phenology might influence growth and how that might vary depending on the phenological stage we are discussing. 
\item Alternatively, something a bit similar that reviews the mechanisms behind each major stage and discusses how they might influence growth.
% I am not convinced that we know enough to get far, but maybe we can outline experiments to test each idea we present?
\end{enumerate}
\item Frederik
\begin{enumerate}
\item (D'Orangeville 2018 paper) I think there is an important lesson (at least for me) that the authors discuss later: Potential effects of drought (but probably also stimuli) towards the end of the GS may be seen only in the next year! In some species shoots are already prebuilt in late summer for next year and reduced water as well as NSC reserves might influence performance in the next year. I think this is also an interesting gap for a future experiment:
\begin{enumerate}
\item 3 Treatments:
\begin{enumerate}
\item Control (normal growing season length with perfect conditions)
\item GS extended (potentially to split up even in spring and autumn extension)
\item GS shortened (drought or cool temp) in autumn (or again to split up an additional treatment in spring)
\end{enumerate}
\item After this treatment during the first GS all saplings experience another GS under the same control conditions. Then big harvest...
\end{enumerate}
\item I could think also of another experiment that focuses more on the faith of C, because it is still unclear to me when C is mainly taken up and where it ends up greatly influencing its residence time: Perhaps a CO$_2$ labeling experiment with different labeling occasions within the GS to evaluate
\begin{enumerate}
\item  How efficiently was the C fixed and
\item  Where is the fixed sugar incorporated (structural growth in different organs, starch reserves, root exudates…)
\end{enumerate}
\end{enumerate}
\item Kavya
\begin{enumerate}
\item Concept paper on climate, phenology, growth – what we know, and what are things the research community should work on perhaps emphasizing the need more more integration of physiology and phenology (and maybe molecular bio)
\item Get all common garden data possible to see if we could get some of the questions answered related to phenology (are there datasets already of phenology studies in common gardens that haven't specifically looked at growth but could be used to do so) and growth (could we core some of these trees to get at overall and interannual growth, at least for radial growth) – what data is present, what is missing, can we supplement missing data, what species can we look at?
\item Definitely also interested in seeing where the carbon is going (like Fredi) in years of longer or earlier growing season if not to radial growth although it would be hard for us to do this one
\item Flux tower work inspired by and complementary to work that has been done already perhaps: When does growth actually happen within the growing season - we could use flux tower data (though I have no idea how to even begin looking at this kind of data) to look at when in the year carbon drawdown is happening – if earlier spring phenology means earlier carbon drawdown then even if there isn’t tree radial growth, net carbon uptake is a good thing, maybe not for sequestration but short term drawdown? It would be cool to see if precipitation and vpd limitations later in the growing season lead to lower C uptake
\end{enumerate}
\item Janneke
\begin{enumerate}
\item Outline a review / synthesis / future directions paper that specifically focuses on when and how changes in phenology influence tree growth, and where that carbon goes. From the papers I've read I think there are few that integrate insights and measurement techniques from all subfields (phenology, tree ring, ecosystem / climate change), and a surprising number of unknowns. 
\item I think while we are doing this we should outline the perfect observational study and experiments to illustrate key unknowns. This will be good for the paper (more interesting), but also will help us decide whether any of us (in whatever combination of people) will want to start such a study / observational data set / analysis.    
\end{enumerate}
\item Lizzie
\begin{enumerate}
\item Important question -- dendro people who do temperature reconstructions with ring width, do they use the same year? (Meaning -- their whole field correlates the year with the climate??!!)
\item Review paper integrating community, ecology, dendro and physio perspective (so we skip the flux/remote sensing people) ... Title is something like `Chasms in tree growth' and I am working on a theme song.
\item Figure out the ideal 1-2 year greenhouse experiments to do to nail down variation across species in phenology x growth (x climate?) relationships ....
\item Using elevational tree ring data look both at elevational trends in ring width and trends over time (bracket variation possible and compare to Dow et al. or such? Basically get some numbers here ...).
\item Compare elevational trends in ring with with provenance trends (from common gardens)?
\end{enumerate}
\item Ailene
\begin{enumerate}
\item Review/synthesis paper on relationship between phenology, growing season length, growth and carbon storage in trees:
\begin{enumerate}
\item (When) Are earlier growth onset years associated with longer growing season length?
\item  (When) is longer growing season length associated with more growth
\item  What is the relationship between annual growth (i.e. ring width) and carbon storage?
\item What are the implications of all of this for climate change and forest carbon storage- aka “natural climate solutions” (i.e., what feedbacks can we expect between tree carbon storage under future warming)?
\end{enumerate}
\item  Review to answer “Do longer growing seasons result in more growth? Why/why not?”
\begin{enumerate}
\item What do we learn from different sources and what does this mean?
\begin{enumerate}
\item Experiments
\item Latitudinal and elevational studies
\item Time series
\end{enumerate}
\item How do patterns vary in space versus time and what does this mean for forecasting patterns under future climate change?
\end{enumerate}
\item Perspectives paper focused on “natural climate solutions” and whether/how much incorporating feedbacks between climate and vegetation responses might alter estimates of carbon storage
\begin{enumerate}
\item What is the magnitude of variation in growth across long vs. short growing season?
\item Compare estimates in above vs. below ground growth and carbon storage
\item    what is the relationship between tree carbon storage and ecosystem carbon storage (i.e. if an individual or populations/community of trees stores more carbon in a given year or under certain conditions, does this tell us anything about the carbon stored in the above and below ground parts of the ecosystem?
\end{enumerate}
\end{enumerate}
 \item Cat           
\begin{enumerate}
\item {\bf Most reasonable:} write up a concept paper that summarizes our discussions and findings these last few weeks. Greatest surprise being little overlap across disciplines and the assumption that there is a stronger relationship between phenology and tree ring growth than we actually have evidence for
\item {\bf More exciting to me:} I'd love to get into the nitty gritty of exactly that relationship---phenology and C sequestration. I would love to do a synthetic review of understanding these linkages! There is ample evidence of the effects of climate change on phenology, but do these relationships hold for C sequestration? And what do our findings suggest for implementing conservation management practices in the future, especially Improved Forest Management practices?
\end{enumerate}
\end{enumerate}

{\bf Week 8: Paper round robin}\\
These are the notes for those who did not make it, other papers discussed quickly at the start of meeting... 
\begin{enumerate}
\item (Frederik): My brief summary for the paper D'Orangeville, L. \emph{et al.} (2018):
Trees across eastern Nordamerica are most sensitive to drought around the summer solstice during peak radial growth activity. Total growth was therefore mainly determined by the conditions during this time (Mai-July) and barely at the end of the GS (growing season).
\item (Cat): Here are some very brief notes on my paper (Clark et al., 2014):
\begin{enumerate}
\item Compared phenology at Duke Forest (North Carolina) and Harvard Forest (Massachusetts)
\item     Assessed mean annual temp, mean spring temp, VPD, soil moisture
\item     Growth Chamber experiment and developed a continuous development model using a hierarchical, state-space framework
 \item    Major Findings:
 \begin{enumerate}
 \item     Late winter temperatures are the strongest predictors for BB for both sites and should be the most accurate predictive tool for future phenology
 \item       VPD and soil moisture did not play a major role for these two sites
 \item       There was a very weak relationship for MAT and budburst; late winter/early spring are the most useful temperatures
 \item       The CDM framework better captures variation throughout the year, which will be useful for making future predictions of phenology
\end{enumerate}
\end{enumerate}
\end{enumerate}



\end{document}



\bibliography{..//refs/micro}
